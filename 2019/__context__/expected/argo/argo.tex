\usemodule[pycon-yyyy]
\starttext

\Title{Testowanie mikroserwisów z Argo i Zalenium}
\Author{Maciej Brzozowski}
\MakeTitlePage

\subsection[wprowadzenie]{Wprowadzenie}

W artykule zostaną przedstawione problemy i ich rozwiązania, które
napotkaliśmy w projekcie zbudowanym w architekturze mikroserwisów oraz
bardzo wysokiej gęstości zmian i zastosowanych narzędzi, przyjrzymy się
jak doszlśmy do zaawansowanej automatyzacji. Tekst stanowi zarys tego,
co zostanie pokazane na prezentacji. Po przeczytaniu tych kilku
akapitów, widzowi będzie dużo łatwiej zrozumieć, co właściwie dzieje się
(wysokopoziomowo) na ekranie.

\subsection[testowany-produkt]{Testowany produkt}

Finalny produkt w założeniu ma pozwalać na przeszukiwanie przepastnych
baz wiedzy w banku Nordea i umożliwienie znalezienia źródeł potrzebnych
informacji oraz uzyskanie dostępu do danych. Wszystko opakowane w piękny
interfejs i obsługiwane w prosty sposób.

\section[technologia]{Technologia}

Produkt łączy w sobie rozproszone aplikacje napisane w języku Scala oraz
część graficzną napisaną z wykorzystaniem frameworku AngularJS.
Aplikacje komunikują się za pomocą protokołu HTTP oraz wykorzystują
Kafkę jako kolejkę komunikacyjną. Elementy budowane są za pomocą
środowiska Bazel by w łatwy i szybki sposób skompilować różne
technologie do postaci obrazów Dockera oraz utworzyć potrzebne definicje
zasobów dla środowiska zarządzania kontenerami Openshift/Kubernetes. Do
tego dochodzą systemy zewnętrze, do których system ma się zintegrować.

\subsection[napotkany-problem]{Napotkany problem}

Projekt składa się z dość sporej liczby rozproszonych serwisów, gdzie
każdy z nich jest w postaci kontenera. Postawienie całego systemu wraz z
jego konfiguracją nie jest trywialnym zadaniem. W celu przetestowania
systemu postanowiliśmy skupić się na testowaniu tylko tych serwisów,
które się ze sobą bezpośrednio komunikują jaki i już wystawionej całej
aplikacji. Opcja druga pozwala na sprawdzenie scenariuszy od końca do
końca jak i potwierdzenia czy aktualny stan konfiguracji jest należyty.
Jako, że system docelowo znaduje się na platformie Openshift, testy od
końca do końca są rownież wykonywane na tej platformie. Przy takich
testach wykorzystany jest gotowy sposób wystawiania systemu przygotowany
przez zespół DevOps. Po stronie testera leży tylko przygotowanie
scenariuszy testowych. Natomiast w sytuacji kiedy testowane są powiązane
serwisy występuje potrzeba przygotowania fragmentów systemu, gdzie
występuje grupa powiązanych serwisów wraz z ich konfiguracjami. Tym
razem tester musi przygotować środowisko testowe, w którym możne
zasymulować rożne przypadki testowe i oczywiście napisać scenariusze
testowe, które sprawdzą interakcje pomiędzy tymi serwisami. Nastała
potrzeba wykorzystania narzędzia, które pomoże w zestawieniu grupy
kontenerów. Nasuwającym się remedium na taki problem jest zastosowanie
Docker Compose, który szybko i prosto umożliwi zastawienie grupy
kontenerów. W praktyce zastosowanie Docker Compose wprowadza redundantny
sposób utrzymania konfiguracji dla każdego serwisu, gdyż występuje
potrzeba przygotowania konfiguracji dla środowiska Openshift i Docker
Compose. Docker Compose nie posiada również możliwości budowania
przepływów pracy. Postawione raz kontenery są skonfigurowane
\quotation{na sztywno} i nie można symulować przypadków testowych w
prosty sposób. Wystapiła potrzeba narzędzia, które pozwoli budować
przepływy pracy oraz pozwoli na użycie gotowych już konfiguracji ze
środowiska Openshift/Kubernetes. Odpowiedzią na zaistniały problem
okazało się narzędzie o nazwie Argo, które pozwala na budowanie
przepływów pracy w środowisku Kubernetes.

\subsection[testowanie]{Testowanie}

Aplikacja testowana jest na różnych poziomach, począwszy od testów
jednostkowych, systemowych i testach od końca do końca. Testy
jednostkowe są pisane przez programistów danej funkcji systemu. Testy
systemowe i testy od końca do końca są pisane przez testerów
dedykowanych do projektu.

\subsection[użyte-narzędzia]{Użyte narzędzia}

Do testów jednostkowych wykorzystana jest biblioteka Scalatest, która
jest naturalnym wyborem dla testów jednostkowych dla języka Scala. Dla
wyższych poziomów testowania należało wybrać narzędzie, które umożliwi
czytanie testów dla osób nietechnicznych i będzie w prosty sposób
rozszerzalne - te założenia świetnie spełnia Robot Framework. Ze
względów wymienionych w jednym z wcześniejszych akapitów do budowania
przepływów pracy użyty został Argo. Dla przeglądarkowych testów
graficznych oczywistym wyborym jest Selenium, lecz w tym przypadku
postawiliśmy na projekt Zalenium, który rozszerza funkcje tego
pierwszego.

\section[robot-framework]{Robot Framework}

Robot Framework jest oprogramowaniem napisanym w Pythonie, które
umożliwia łatwą automatyzację zadań przy użyciu tzw. keywordów, które
pozwalają ułożyć kod w bloki proste do zrozumienia dla
niewtajemniczonych w arkana programistycznych czarów.

\section[argo]{Argo}

Argo zaś jest narzędziem do budowania przepływów zadań i procesów w
środowisku Kubernetes. Argo aranżuje przepływ pracy w postaci
kaskadowych kroków lub też acyklicznego grafu. Kazdy krok schematu jest
osobnym kontenerem. Przy jego pomocy udało się zbudować scenariusze
testowe w docelowym środowisku (Openshift/Kubernetes), w którym
dostarczana jest aplikacja.

\section[zalenium]{Zalenium}

Jest to wyprodukowane przez Zalando rozszerzenie Selenium Grid, który
pozwala zestawić wiele procesów automatycznych testów Selenium w postaci
kontenerów, na różnych przeglądarkach i z podglądem na żywo. Zalenium
jest wykorzystywane do testowania aplikacji od strony graficznej.

\subsection[jak-to-zadziałało]{Jak to zadziałało?}

Wszystkie procesy wystawieniowe zostały zaprojektowane przy pomocy Argo.
Wystarczyło do tego podjąć kilka dodatkowych kroków, w postaci
uruchomienia obrazu Zalenium dla testów graficznych, puszczenia testów
czy klonowania repozytorium testowego. Po podaniu odpowiednich
argumentów, wszystko wykonuje się samo. Testerowi pozostaje rozszerzanie
pakietów testowych i raportowanie defektów.

\subsection[źródła]{Źródła}

\startitemize[n,packed][stopper=.,width=2.0em]
\item
  Robot Framework, \useURL[url1][https://robotframework.org]\from[url1]
\item
  Argo, \useURL[url2][https://argoproj.github.io]\from[url2]
\item
  Selenium, \useURL[url3][https://www.seleniumhq.org]\from[url3]
\item
  Zalenium,
  \useURL[url4][https://opensource.zalando.com/zalenium]\from[url4]
\item
  Docker, \useURL[url5][https://www.docker.com]\from[url5]
\item
  Docker Compose,
  \useURL[url6][https://docs.docker.com/compose]\from[url6]
\item
  Openshift, \useURL[url7][https://www.openshift.com]\from[url7]
\item
  Kubernetes, \useURL[url8][https://kubernetes.io]\from[url8]
\item
  Scala, \useURL[url9][https://www.scala-lang.org]\from[url9]
\item
  Scalatest, \useURL[url10][http://www.scalatest.org]\from[url10]
\item
  AngularJS, \useURL[url11][https://angularjs.org]\from[url11]
\item
  Kafka, \useURL[url12][https://kafka.apache.org]\from[url12]
\item
  Bazel, \useURL[url13][https://bazel.build]\from[url13]
\stopitemize


\stoptext
