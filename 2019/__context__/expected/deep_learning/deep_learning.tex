\usemodule[pycon-yyyy]
\starttext

\Title{100\letterpercent{} win rate?! A gentle introduction to (Deep) Reinforcement Learning}
\Author{Wadim Sokołowski}
\MakeTitlePage

\section[abstract]{Abstract}

Machine Learning is so pervasive today that people use it dozens of
times every day without even knowing it. Many researchers think that
this is the best way to make progress towards human-level AI.

With this article I'm aiming to make the readers familiar with probably
the least known, but in my opinion the most exciting area of Machine
Learning - the Reinforcement Learning. I'm going to cover the basic
concepts and ideas of Reinforcement Learning, explain how it can be used
together with Deep Learning and discuss possible applications of such
technique in the real life.

\section[introduction]{Introduction}

Recent boom in Machine Learning proved that artificial intelligence can
have its applications in almost every area of today's human life --
education, health, finances, autonomous driving and many others.

While neural networks are responsible for latest breakthroughs in
various areas like computer vision, machine translation and speech
recognition, they can also be combined with Reinforcement Learning
algorithms to create outstanding solutions. Examples may include
DeepMind's AlphaStar system beating the world's best players at the
StarCraft II game or Deep Q Network, an algorithm capable of competing
with human experts in playing numerous Atari video games.

\section[reinforcement-learning]{Reinforcement Learning}

Reinforcement Learning is a Machine Learning method having its roots in
the early 1980's. It's an approach where an agent learns how to behave
in an environment by performing actions and seeing the results. Trial
and error search and delayed reward are the most relevant
characteristics of Reinforcement Learning. This method allows machines
and software agents to automatically determine the ideal behavior within
a specific context in order to maximize its performance. Simple feedback
(usually called reward or reinforcement signal) is required for the
agent to learn which action is best for the current state.

Learning from interaction with the environment comes from human's
natural life experiences -- a robot will learn how to walk in the same
way as the little child learns to take it first steps.

The basic idea behind Reinforcement Learning is visualized on below
figure:

\placefigure{}{\externalfigure[rl_setup.png][width=7cm]}

\section[key-components]{Key components}

Except for the agent and the environment, there are few other key
sub-elements of Reinforcement Learning system:

\startitemize[packed]
\item
  Policy
\item
  Reward/reinforcement signal
\item
  Value function
\item
  Model
\stopitemize

A {\bf policy} defines the learning agent's way of behaving at a given
time. In other words, a policy is a mapping from perceived states of the
environment to actions to be taken when in those states. In easiest
cases, the policy may be a simple function or lookup table, whereas in
the more complicated ones it may involve extensive computation such as a
search process. The policy is the core of a Reinforcement Learning agent
in the sense that it alone is sufficient to determine agent's behavior.

A {\bf reward/reinforcement signal} defines the goal in a Reinforcement
Learning problem. On each time step, the environment sends to the agent
a single value called the reward. The agent's objective is to maximize
the total reward it receives over the long run. The reward signal thus
describes what the good and bad events are for the agent in specific
environment. The reward signal is the primary basis for altering the
policy - if an action selected by the policy is followed by low reward,
then the policy may be changed to select some other action in that
situation in the future.

Whereas the reward signal is a short-term indication of what is good in
an immediate sense, a {\bf value function} specifies what is good in the
long run. In general, the value of a state is the total amount of reward
an agent can expect to accumulate over the future, starting from that
state. For example, a state might always yield a low immediate reward
but still have a high value because it is regularly followed by other
states that yield high rewards. Rewards are basically given directly by
the environment, but values must be estimated and re-estimated from the
sequences of observations an agent makes over its entire lifetime. In
fact, the most important component of almost all Reinforcement Learning
algorithms is a method for efficient values estimation. This is
considered to be the hardest part of developing Reinforcement Learning
solutions - researchers are struggling with finding most optimal
solution for many decades.

The last element of Reinforcement Learning systems to mention is a
{\bf model} of the environment. This is a component allowing to make
inferences about how the environment will behave. For example, given a
state and action, the model might predict the resultant next state and
next reward. Models are used for planning, which can be meant as a way
of performing an action by considering possible future situations before
they are actually experienced. Methods for solving Reinforcement
Learning problems that use models and planning are called model-based
methods. Opposed to them, researchers have distinguished simpler
model-free methods that are explicitly trial-and-error approaches.

Above description of how in general the Reinforcement Learning methods
work can be summarized with below sentence:

{\bf At each step t, agent executes action At, receives reward Rt, makes
observation Ot, and infers state St.}

\placefigure{}{\externalfigure[rl_setup2.png][width=7cm]}

\section[combining-reinforcement-learning-with-deep-learning]{Combining
Reinforcement Learning with Deep Learning}

In Supervised Learning, Deep Learning is used to prevent
hand-engineering of features for unstructured data such as images or
text. In Reinforcement Learning, Deep Learning is used largely for the
same reason. With neural networks, Reinforcement Learning problems can
be tackled without need for much domain knowledge. To exemplify this, a
well-known Atari 2600 Pong game can be considered.

\subsection[playing-pong-with-deep-reinforcement-learning]{Playing
Pong with deep Reinforcement Learning}

In traditional learning, there is a need to explicitly extract features
from the game positions to gain meaningful information. Using neural
networks allows feeding the raw game pixels into the algorithm and
letting it create high-level non-linear representations of the data.

For doing this, a one can construct a policy network that is trained
end-to-end, meaning that the model takes game states as an input and
outputs a probability distribution over possible actions that the agent
takes.

In case of playing Pong game, the action is either going UP or DOWN. An
example setup for solving such task is presented on below figure:

\placefigure{}{\externalfigure[pong.png][width=7cm]}

At first glance above model might look like typical Supervised Learning
setup, for example for image classification task, but the key difference
here is that there aren't any labels for game states given and thus the
network cannot be trained in the same fashion.

\subsubsection[training-process]{Training process}

The training process of an agent playing Pong game can be modeled as a
loop of given steps:

\startitemize[packed]
\item
  Agent receives state S0 from the Environment (in this case it's the
  first frame of the game (state) from Atari 2600 Pong (environment))
\item
  Based on that state S0, agent takes an action A0 (moves UP or DOWN)
\item
  Environment transitions to a new state S1 (new frame)
\item
  Environment gives reward R1 to the agent (in case of Pong, the reward
  function can be designed as \quotation{+1 reward if the ball went past
  the opponent, a -1 reward if the ball was missed, 0 otherwise})
\stopitemize

The above loop outputs a sequence of state, action and reward. The goal
of the agent is to maximize the expected cumulative reward.

\subsubsection[reward-discounting]{Reward discounting}

The cumulative reward at each time step t can be written as:

\placefigure{}{\externalfigure[formula1.png]}

Which is equivalent to:

\placefigure{}{\externalfigure[formula2.png]}

However, in reality, rewards cannot be just added like that. The rewards
that come sooner (in the beginning of the game) are more probable to
happen, since they are more predictable than the long term future
reward. Nevertheless, closer to the end of the episode, agent's actions
probably much harder affect the final result as they determine whether
or not the paddle reaches the ball and how exactly it hits the ball. If
an agent moved UP at the first frame of the episode, it probably had
very little impact on whether or not he won the game.

To tackle the problem described above, the Pong agent ought to learn in
a way that actions taken towards the end of an episode more heavily
influence learning than actions taken at the beginning. This process is
called {\bf reward discounting}.

In order to discount the rewards, a one should define a discount rate
{\em γ} called gamma. It must be a real number between 0 and 1.

\startitemize[packed]
\item
  The larger the gamma, the smaller the discount. This means the
  learning agent cares more about the long term reward. A discount
  factor of 1 would make future rewards worth just as much as immediate
  rewards.
\item
  On the other hand, the smaller the gamma, the bigger the discount.
  This means the agent cares more about the short term reward.
\stopitemize

The discounted cumulative expected rewards can be calculated as:

\placefigure{}{\externalfigure[formula3.png]}

If {\em γ} is 0.8, and there's a reward of 10 points after 3 time steps,
the present value of that reward will be equal 0.8³ x 10.

\subsubsection[exploration-vs-exploitation-dilemma]{Exploration vs
exploitation dilemma}

The exploration-exploitation trade-off is a fundamental dilemma whenever
a human learns about the world by trying things out. The dilemma is
between choosing what he knows and getting something close to what he
expects (\quote{exploitation}), and choosing something he isn't sure
about and possibly learning more (\quote{exploration}). The same dilemma
can be applied to Reinforcement Learning area. In case of exploitation,
the agent maximizes rewards through behavior that is known to be
successful. As opposed to this approach, exploration allows agent to
experiment with novel strategies that may improve rewards returned in
the long run.

\placefigure{}{\externalfigure[exploration-exploitation.png][width=7cm]}

The first question one may ask is: why is exploration needed at all? The
problem can be framed as one of obtaining representative training data
for various Machine Learning setups (e.g., proper set of train and
validation images used with deep convolutional neural network designed
for image recognition). In order for an agent to learn how to deal
optimally with all possible states in an environment, it must be exposed
to as many of those states as possible. However, unlike in traditional
Supervised Learning settings, the agent in a Reinforcement Learning only
has access to the environment through its own actions.

\placefigure{}{\externalfigure[exploration-exploitation2.png][width=7cm]}

When working with Reinforcement Learning algorithms, it's crucial to
both explore and exploit the environment on which the agent is
operating.

\section[some-code]{Some code}

With the below piece of Python code (and some additional modules like
\useURL[url2][https://github.com/keras-rl/keras-rl][][keras-rl]\from[url2])
the agent can learn to solve CartPole task. The goal is to prevent the
pendulum starting upright from falling over. For more details about the
task, please refer to
\useURL[url3][https://github.com/openai/gym/wiki/CartPole-v0][][official
documentation of CartPole-v0]\from[url3] in official OpenAI Gym
toolkit's repository.

\starttyping
import numpy as np
import gym

from keras.models import Sequential
from keras.layers import Dense, Activation, Flatten
from keras.optimizers import Adam

from Reinforcement Learning.agents.dqn import DQNAgent
from Reinforcement Learning.policy import EpsGreedyQPolicy
from Reinforcement Learning.memory import SequentialMemory

# Define the environment (task)
ENV_NAME = 'CartPole-v0'

# Get the environment and extract the number of actions available in
# the Cartpole problem
env = gym.make(ENV_NAME)
np.random.seed(123)
env.seed(123)
nb_actions = env.action_space.n

# Build neural network model

model = Sequential()
model.add(Flatten(input_shape=(1,) + env.observation_space.shape))
model.add(Dense(24, activation='relu'))
model.add(Dense(24, activation='relu'))
model.add(Dense(nb_actions, activation='linear'))

# Use random exploration strategy to enlarge the space explored
policy = EpsGreedyQPolicy()

# Create the buffer for storing the result of actions we performed
# and the rewards we get for each action
memory = SequentialMemory(limit=50000, window_length=1)

# Create Reinforcement Learning (in this case - Deep Q-Learning) agent
dqn = DQNAgent(
    model=model, nb_actions=nb_actions, memory=memory, policy=policy, 
    target_model_update=1e-2, nb_steps_warmup=100
)
dqn.compile(Adam(lr=1e-3), metrics=['mae'])

# Run the training loop
dqn.fit(env, nb_steps=1000, visualize=True, verbose=2)

# Test the reinforcement learning model
dqn.test(env, nb_episodes=20, visualize=True)
\stoptyping

\section[reinforcement-learning-applications]{Reinforcement
Learning applications}

Nowadays, Reinforcement Learning has a wide range of applications.
Numerous solutions are being used in everyday life and many will be
continuously introduced in upcoming years. Some major domains where
Reinforcement Learning can be practically used are as follows:

\subsection[robotics-and-industrial-automation]{Robotics and
industrial automation}

Applications of Reinforcement Learning in high-dimensional control
problems, like robotics, have been the subject of research for many
years and companies are beginning to use Reinforcement Learning to build
products for industrial robotics. A common example is the use of AI for
tuning machines and equipment where expert human operators are currently
being used.

Robot trained with deep Reinforcement Learning approach can learn to
pick a device from one box and put it in a container. Whether it
succeeds or fails, it memorizes the object, gains knowledge and train's
itself to do this job with great speed and precision. Many warehousing
facilities used by eCommerce sites and other supermarkets are using
these intelligent robots for sorting millions of products every day and
helping to deliver the right products to the right people.

\subsection[education]{Education}

Online platforms are beginning to experiment with using Machine Learning
to create personalized experiences. Researchers are investigating the
use of Reinforcement Learning and other Machine Learning methods in
tutoring systems and personalized learning. The use of Reinforcement
Learning can lead to training systems that provide custom instruction
and materials tuned to the needs of individual students.

\subsection[text-speech-and-dialog-systems]{Text, speech and
dialog systems}

Companies collect a lot of text and good tools that can help unlock
unstructured data will cause revolutions in various industries. Deep
Reinforcement Learning can be applied for abstractive text summarization
- a technique for automatically generating summaries of original text
documents.

Reinforcement Learning is also being used to allow dialog systems (i.e.,
chatbots) to learn from user interactions and thus help them improve
over time. Socialbots need to rank possible responses and select the
best by exploring the tradeoff between immediate satisfaction versus
long-term reward of selecting a certain response.

\subsection[media-and-advertising]{Media and advertising}

Reinforcement Learning, thanks to its adaptability to changing
environments, can help with cross-channel marketing optimization and
real time systems for online display advertising. Based on user's
behavior and preferences, a recommendation system agent can learn to
guess user's \quotation{mood} and adjust the advertisement in reasonable
steps.

\subsection[health-and-medicine]{Health and medicine}

The Reinforcement Learning setup of an agent interacting with an
environment receiving feedback based on actions taken shares
similarities with the problem of learning treatment policies in the
medical sciences. Reinforcement Learning can be used to optimize
medication dosing and medical equipment utilization.

\subsection[autonomous-driving]{Autonomous driving}

The self-driving car is a hot topic in the automotive industry. Current
AI solutions for determining appropriate stopping distance, distance
from another vehicle and other data that dramatically decreases the
chances of car accidents have proved to be successful.

The driving policy of autonomous car can be decomposed into two
components:

\startitemize[packed]
\item
  hard code for rule based \quotation{hard constraints}
\item
  Reinforcement Learning system that learns how to adapt to scenarios,
  balance between aggressive/defensive behavior, negotiate with other
  drivers
\stopitemize

The Reinforcement Learning setting can be described as follows:

\startitemize[packed]
\item
  State: sensor input: lanes/cars/pedestrians
\item
  Action: steering wheel/throttle/brake petal
\item
  Reward: drive toward goal without causing accident
\stopitemize

Looking at the recent achievements in automotive driving industry, there
is no doubt that mixture of Reinforcement Learning with Deep Learning
will be the most promising approach to achieve human-level control for
car driving.

{\bf Games}

Games are great testbeds for Reinforcement Learning algorithms. Deep
Learning can be used in conjunction with existing Reinforcement Learning
techniques to play Atari games, beat a world-class player in the game of
Go or solve complicated riddles. Deep Learning has been shown to be
successful in extracting useful, nonlinear features from
high-dimensional media such as images, text, video and audio.

The AlphaGo system created by artificial intelligence company Google
DeepMind achieved super-human level in playing Go game, an abstract
strategy board game for two players, where there are about $250^150$
possible game states. The system uses Reinforcement Learning only,
without any human knowledge.

The Reinforcement Learning setting for playing Go can be described as
follows:

\startitemize[packed]
\item
  State: game board
\item
  Action: put stone
\item
  Reward: win or lose at the end
\stopitemize

The exhaustive search is infeasible, as the search space is too big.
Breadth of the search is reduced by sampling actions from a policy
(possible best moves in current position), and depth of the search is
reduced by value function (position evaluation).

Another interesting mention is OpenAI Dota-2 bot designed to play
Multiplayer Online Battle Arena (MOBA) video game Dota 2. Usage of
Reinforcement Learning techniques allowed the AI to discover physical
skills like tackling, ducking, faking, kicking, catching, and diving,
which resulted in beating the world's top professionals at one versus
one matches in August 2017.

\section[sources]{Sources}

\startitemize[n,packed][stopper=.]
\item
  Sutton, R. S. and Barto, A. G. Reinforcement Learning: An
  Introduction. MIT Press, Cambridge, MA, 2017.
\item
  \quotation{Playing Atari with Deep Reinforcement Learning}.
  arXiv:1312.5602 {[}cs{]}, December 2013. arXiv.org,
  \useURL[url4][http://arxiv.org/abs/1312.5602]\from[url4].
\item
  OpenAI Gym hompeage, \useURL[url5][https://gym.openai.com/]\from[url5]
\item
  DeepMind, AlphaGo homepage,
  \useURL[url6][https://deepmind.com/research/alphago/]\from[url6]
\stopitemize


\stoptext
