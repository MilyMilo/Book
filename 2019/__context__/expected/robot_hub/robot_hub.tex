\usemodule[pycon-yyyy]
\starttext

\Title{RfHub2 -- od testera dla testera. Dokumentacja w jednym miejscu}
\Author{Maciej Wiczk}
\MakeTitlePage

\subsection[wprowadzenie]{Wprowadzenie}

W pracy nad kodem, który wykonuje się z tzw. ręki, to jest niesterowanym
odgórnym systemem, zespoły często napotykają problem powtarzania się. W
dużej bazie funkcjonalności często dłużej trwa odnalezienie istniejącej
implementacji, niż stworzenie jej na nowo, stąd potrzeba użycia narzędzi
indeksujących i samotworzącej się dokumentacji, która \quotation{na
żywo} będzie dostępnej dla każdego użytkownika wspólnego systemu.

\subsection[napotkane-problemy]{Napotkane problemy}

\section[robot-framework]{Robot Framework}

Na początku całej drogi w zespole testowym, zdecydowaliśmy się
wykorzystać Robot Framework, jako bazę do automatyzacji zadań testowych
i ogólnie zadań pracowniczych. Robot Framework cechuje się prostym
składniowo podejściem do \quotation{programowania}, w wyniku którego
powstają - najlepiej rozczłonkowane - procedury, które są w prosty
sposób zrozumiałe przez użytkowników na ogół niezwiązanych z informatyką
czy w ogóle językiem technicznym.

\section[niespotykane-przypadki-użycia]{Niespotykane przypadki
użycia}

Okazało się jednak, że użycie tego konkretnego frameworka generuje pewną
potrzebę. Tą potrzebą jest tworzenie przybudówek do Robot Frameworka,
związanych z tym, że nikt wcześniej nie używał go konkretnie do
implementacji zadań automatyzujących związanych z Apache Hadoopem,
Oozie, czy Hadoop File Systemem (HDFS).

\section[nadaktywny-zespół]{Nadaktywny zespół}

Nasz zespół napotkał problem \quotation{pokusy} tworzenia kodu. Wynikało
to z prostej zależności -- zatrudniliśmy masę fascynatów programowania,
po czym powiedzieliśmy im: nie programujcie. Skończyło się
zniechęceniem, bo często i gęsto napotkane problemy nie znajdowały ani
gotowego rozwiązania (trudne do odnalezienia w dużej bazie kodu) ani
implementacji (DRY - Don't Repeat Yourself).

\section[sprzedaż-rozwiązań-reszcie-firmy]{\quotation{Sprzedaż}
rozwiązań reszcie firmy}

Ten problem nasilił się wraz z pojawieniem się kolejnej potrzeby.
Okazało się, że nasze narzędzia potrzebne są nie tylko w naszym zespole,
ale również poza nim. Jak \quotation{sprzedać} to co mamy, skoro sami
nie wiemy do końca co napisaliśmy? Taki stan rzeczy jest absolutnie
nieakceptowalny w wysoce sformalizowanym otoczeniu, jakim jest bank.
Bezsilnie obserwowaliśmy, jak siostrzane zespoły tworzyły podobne
rozwiązania, tracąc zasoby na rzecz powtarzania wykonanej już pracy.

\subsection[robot-framework-hub]{Robot Framework Hub}

\section[poprzednie-próby-rozwiązań]{Poprzednie próby rozwiązań}

Próbowaliśmy wielu usprawnień, zaczynając od oczywistej - tworzenia
dokumentacji. Nie zawsze był jednak czas, by ją uaktualniać; nie zawsze
był czas by ją czytać. Podjęliśmy się organizacji spotkań
międzyprojektowych i międzyzespołowych, ale nie każdy miał ochotę w nich
uczestniczyć i nie zawsze poruszany był akurat ten temat, który
zapobiegłby tworzeniu istniejących rozwiązań. Wtem, jeden z kolegów
zaprezentował nam Robot Framework Hub.

\section[hub]{Hub}

To oprogramowanie open source autorstwa Bryana Oakleya z Oklahomy.
Brayan Oakley stworzył dość proste w założeniu narzędzie, które
analizowało bazę kodu, otrzymaną na wejściu i wyciągało z tej bazy
wszystkie funkcje z bibliotek w pythonie i Robot Frameworku. Wyciągnięte
funkcje są analizowane pod kątem dokumentacji i treści, a z zebranych
danych tworzona jest prosta witryna internetowa, wzorowana na
oryginalnej dokumentacji Robot Framework.

Wydawałoby się, że rozwiązało to wszystkie problemy, aż do momentu,
kiedy w użytkownicy zaczęli zgłaszać błędy w witrynie. Wiele bibliotek
nie zostało odczytanych i brakowało dokumentacji. Wróciliśmy w dużej
mierze do punktu wyjścia.

Niestety, okazało się, że Bryan Oakley zarzucił projekt 3 lata temu i to
w stanie niezgodności z najnowszą wersją Pythona -- Pythonem 3. Z tego
powodu biblioteki korzystające z wielu udogodnień nowej wersji naszego
ulubionego języka programowania, nie były odczytywalne przez RF Hub.

\subsection[robot-framework-hub-2]{Robot Framework Hub\ldots{} 2?}

\section[po-co-to-zrobiliśmy]{Po co to zrobiliśmy?}

Sprawa stała się poważna, mieliśmy gotowe rozwiązanie, które nic nam nie
daje. Zdecydowaliśmy sie na naprawę, ale nasze podejścia nie były
satysfakcjonujące, a to kłopoty z wystawieniem aplikacji, a to jakaś
biblioteka się buntowała. Wtedy naszedł nas inny pomysł: trzeba
ukierunkować niezaspokajalną potrzebę rozwijania aplikacji u naszych
kolegów. Dwóch zdecydowało się na implementację rozwiązania i
udostępnienie go open source dla wszystkich użytkowników Robot
Frameworka.

\section[dlaczego-python-3]{Dlaczego Python 3?}

Wybór platformy był dość oczywisty, padło na Python 3, gdyż jest to
główny język używany w naszym środowisku. Tworzenie rozwiązania w
poprzednim Pythonie nie miałoby sensu, bo to już istnieje. Tworzenie go
w innym języku niż Python byłoby niecelowe, bo Robot Framework jest
głównie narzędziem Pythonowym.

\section[ukierunkowana-nadaktywność]{Ukierunkowana nadaktywność}

Ukierunkowaliśmy nadaktywność recenzentów i twórców oprogramowania w
kierunku stworzenia konkretnego narzędzia. Uznaliśmy to za dobry
kierunek jeśli chodzi o celowość (potrzebne w naszej firmie),
\quotation{marketing} (konkretne rozwiązanie, którym możemy się chwalić
na GitHubie) i zabawę (po prostu fajnie jest coś takiego stworzyć).

\section[zastosowane-rozwiązania]{Zastosowane rozwiązania}

Chcieliśmy, żeby aplikacja była zaprojektowana zgodnie z wszelkimi
dobrymi zasadami, więc zaimplementowaliśmy ORM, przy użyciu SQL Alchemy,
co pozwala na szybką i bezpieczną komunikację aplikacji z bazą danych,
co usprawnia proces zapisywania i usuwania danych z bazy. Ponadto
możliwe jest skorzystanie z innej, niż domyślna baza danych w postaci
SQLite. PostgreSQL, MySql, Sql Server czy Oracle wymagają jedynie
doinstalowania odpowiednich sterowników i podania likalizacji bazy
dnaych. Sercem aplikacji jest framework FastAPI, będący obecnie jednym z
najszybszych pythonowych frameworków tego typu, dodatkowo serwującym
interaktywną dokumentację w standardzie OpenAPI, znaną wcześniej jako
Swagger. Aplikację docelowo stawiamy jako kontener dockerowy, co pomaga
w restarcie i łatwym rozpowszechnianiu tego narzędzia. Cały projekt
używa też TravisCI, żeby umożliwić szybkie i celne testowanie modułów i
integracji między nimi. Nowa wersja programu również wykorzystuje
izolację zadań: część odpowiedzialna za API, komunikację z bazą danych i
frontend, działa niezależnie od modułu obsługujecego analizę bazy kodu i
wysyłki danych do aplikacji. Z drobnych usprawnień, można również nie
uwazględniać podstawowych bibliotek Robot Frameworka, żeby nie zaśmiecać
dużej bazy dodatkowymi, znanymi wszystkim bibliotekami, dodawać nowe
biblioteki bez potrzeby usuwania starych, a takze aktualizację danych
poprzez API.

\subsection[źródła]{Źródła}

\startitemize[n,packed][stopper=.]
\item
  Robot Framework, \useURL[url1][https://robotframework.org/]\from[url1]
\item
  Fast Api, \useURL[url2][https://fastapi.tiangolo.com/]\from[url2]
\item
  SQL Alchemy,
  \useURL[url3][https://docs.sqlalchemy.org/en/13/]\from[url3]
\item
  Repozytorium Github RFHUB,
  \useURL[url4][https://github.com/boakley/robotframework-hub]\from[url4]
\item
  Repozytorium GitHub RFHUB2,
  \useURL[url5][https://github.com/pbylicki/rfhub2]\from[url5]
\item
  DRY Principle - wiki,
  \useURL[url6][https://pl.wikipedia.org/wiki/DRY]\from[url6]
\stopitemize


\stoptext
