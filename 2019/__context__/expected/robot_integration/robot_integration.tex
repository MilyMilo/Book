\usemodule[pycon-yyyy]
\starttext

\Title{Integracja zadań testowych przy użyciu Robot Framework}
\Author{Mateusz Kotas}
\Author{Krzysztof Synak}
\MakeTitlePage

\subsection[wprowadzenie]{Wprowadzenie}

Najgorszą częścą pracy w korporacji jest dla wielu kwestia papierologii
i biurokratyzacji. Pewnie, fajnie jest potestować, a jeszcze fajniej coś
zepsuć; sprawić, że jakość się poprawiła. Ale zawsze dochodzimy do
momentu, w którym należy rozliczyć się z postępów. A to
\quotation{pozamykać zadania}, a to dostarczyć kluczowe wskaźniki
wydajności; a to po prostu stworzyć raporty testowe. W pewnym momencie
jednak dochodzimy do wniosku, ze wszystko to są powtarzalne czynności,
które równie dobrze może wykonać robot.

A tester może się skupić na testowaniu.

\subsection[problem]{Problem}

Po zatrudnieniu całej gamy barwnych i zdolnych ludzi, zauważyliśmy
narastającą ich niechęć do \quotation{klepania}. Klepanie definiowaliśmy
na wiele sposobów, czy to klepanie raportów testowych, czy klepanie tych
samych testów. Skończył się w końcu natłok zadań związany z budowaniem
nowych narzędzi, zaczęło się {\em zwyczajne życie}.

\section[powtarzalne-nudne-zadania]{Powtarzalne, nudne zadania}

Nordea Bank jest silnie sformalizowaną firmą, która stara się zażegnać
biurokratyczne podejście, jednakże póki co, fakty są faktami -- papiery
muszą się zgadzać! Plany testowe, raporty testowe, składowanie
przypadków testowych\ldots{} wszystkie testalia zostaly już omówione i
zarządzone. Pozostaje tylko wypełnić. To nie są zadania dla młodych,
ambitnych i żywych specjalistów.

\section[złe-rozłożenie-aktywności]{Złe rozłożenie aktywności}

Jak to zwykle bywa w papierologicznych środowiskach, okazało się, że
testerzy wykonują kawał dobrej roboty\ldots{} który nie jest testami.
Wystawianie nowych wersji produktów zaczęło się opóźniać, błędy
wykrywane dopiero po fakcie, ale według papierów wszystko jest
perfekcyjne.

\section[niewyróżniające-się-na-rynku-stanowisko]{Niewyróżniające
się na rynku stanowisko}

Przez powyżej opisane sytuacje, szybko okazało się, że nawet rekrutacja
nie do końca wykazuje się sukcesami. Uczciwe stawianie sprawy dotyczącej
biurokracji i zadań testera odstręczało niektórych, a innych nie
zachęcało -- przecież takich miejsc pracy jest wiele.

\subsection[rozwiązania]{Rozwiązania}

Zdecydowaliśmy się wykorzystać narzędzia obecne i nieobecne dotąd w
Nordei do zautomatyzowania i uatrakcyjnienia naszej pracy i wyników
naszej pracy. Jak wiadomo, nie tylko dane się liczą, ale również
czytelna i atrakcyjna ich prezentacja. Do osiągnięcia tego celu
zaprzęgnęliśmy wszystkich naszych inżynierów. Zmieniliśmy język używany
do określania naszego zespołu. Po wydaniu pierwszego narzędzia,
oficjalnie już nie byliśmy testerami, tylko {\em inżynierami jakości}.

\section[automatyzacja-za-wszelką-cenę]{Automatyzacja za wszelką
cenę}

Zdecydowaliśmy się podejść do sprawy kompleksowo. Każde zadanie, które
trzeba było wykonać więcej niż raz, w ten sam sposób, sprowadziliśmy do
postaci algorytmu. Każdy dokument opisaliśmy przy pomocy jinja2. Żeby
wszystkie te automatyzacje były dostępne, stworzyliśmy portal testera,
prezentujący proste wizualizacje danych oraz pozwalający na wywołanie
funkcji samo-pracujących.

\section[szeroki-wybór-narzędzi]{Szeroki wybór narzędzi}

\quotation{Rynek}, o ile możemy tak nazwać zbiór narzędzi dostępnych za
darmo dla każdego użytkownika, oferuje ogromną liczbę narzędzi, które
można -- a czasem trzeba -- wykorzystać do automatyzacji procesów w
firmie. Narzędzia zarówno płatne i te bezpłatne, roznią sę często tylko
odrobiną pracy koniecznej do włożenia. Do wdrożenia procesów ciągłej
integracji, można użyć wielu bezpłatnych narzędzi, które w niczym nie
ustępują płatnym opcjom. GitLabCI, TeamCity, Jenkins są równie użyteczne
co Bamboo i GitLab. Można wykorzystywać Bitbucket od Atlassiana, a można
po prostu postawić instancję Gita. Można użyć SonarQube, ale też można
bez problemu oprogramować darmowe, proste narzędzia badające pokrycie
kodu.

\section[rozwój-własnych-narzędzi]{Rozwój własnych narzędzi}

Ale to nie koniec! Przecież jako programści -- tym bardziej Pythona --
możemy tworzyć własne narzędzia. Możemy tworzyć narzędzia do obsługi
tych narzedzi. Niejako rekurencyjnie tworząc karuzelę automatyzacji,
można osiągnąć cel najwłaściwszy dla czasów czwartej rewolucji
przemysłowej - ograniczenia ludzkiego wkładu do pracy kreatywnej. Przy
odpowiednim nakładzie pracy na etapie projektowania tych narzędzi, może
się okazać, że bardzo wielu aktywności nie trzeba monitorować ręcznie.
Pokrycie historyjek, informacja o koniecznych przeglądach, badanie
jakości kodu\ldots{} to wszystko da się zrobić automatycznie przy użyciu
własnych narzędzi bądź własnych konfiguracji. Portale informacyjne,
proste aplikacje okienkowe; często przy niedużym nakładzie pracy można
otrzymać narzedzia pożądane w firmie nawet poza naszym zespołem.

\subsection[wynik-działań]{Wynik działań}

Po tych wszystkich wzniosłych hasłach, przyszedł czas na weryfikację
aktywności automatyzacyjno-wytwórczych. Zainwestowawszy czas
\quotation{nieużytków}, czyli kiedy tak czy siak nie była potrzebna
praca okołoprojektowa, stworzyliśmy zestaw narzędzi, które w niedalekiej
przyszłości poskutkowały znaczącymi oszczędnościami pieniędzy i zyskami
w jakości i zadowoleniu tak klientów wewnętrznych jak i nas jako
wytwórców.

\section[zadowolony-zespół]{Zadowolony zespół}

Zespół testerski szybko odnalazł sie w nowej sytuacji. Wkrótce samo
wytwarzanie narzędzi testowych/automatyzujących stało się jedną z zanęt
do pracy z nami. Mając zawsze w niedalekiej perspektywie możliwość
zaprezentowanie czegoś wytworzonego własnymi rękami usprawniło i umiliło
same prace testerskie, zachęcając do tworzenia coraz bardziej
użytecznych testów i narzędzi okołotestowych.

\section[zadowolony-klient]{Zadowolony klient}

W związku ze zwiększonym zadowoleniem testerów i zwiększeniem ich
wydajności, zadowoleni stali się także nasi klienci, właściciele
produktów i kierownicy projektów zauważyli, że po zmniejszeniu nakładu
czasowego na wytwarzanie papierów, sama treść tych papierów stała się
dużo lepsza, obszerniejsza i bardziej celowa.

\section[narzędzia-rozprzestzenione-na-cała-firmę]{Narzędzia
rozprzestzenione na cała firmę}

Nie minęło dużo czasu, nim zaczęli nas odwiedzać pracownicy z całkiem
innych działów firmy. Byli zainteresowani bazą bibliotek, narzędzi i
mechanizmów automatyzacyjnych. Zjawiali się ze swoimi pomysłami, ale i
pomagali - i pomagają - rozwijać istniejące narzędzia. Stworzony przez
nas portal do generacji raportów i planów testów, wykorzystujący dane z
Jiry, został przyjęty do realizacji jako narzędzie obowiązujące w całej
firmie, wraz z przydzieleniem konkretnego budżetu na rozwój.

\section[przyśpieszenie-dostarczania-i-podwyższenie-jakości]{Przyśpieszenie
dostarczania i podwyższenie jakości}

Dzięki zautomatyzowanym procesom dostępnym na tzw. {\em klik}, znacząco
zwiększyła się przepustowość zespołu testerskiego. Mimo, że część czasu
każdego z inżynierów była poświęcona na rozwój wspólnych narzędzi, i tak
odnotowano wzrost liczby znalezionych błędów i usprawnień w projektach.
Testy przestały opóźniać publikację nowych wersji oprogramowania, bo
znikła potrzeba \quotation{klepania} raportów co sprint.

\section[firma-nie-odnotowała-kosztu]{Firma nie odnotowała kosztu}

A co najważniejsze w tym wszystkim, firma nie odnotowała żadnego
(dodatkowego) kosztu. Wykorzystane narzędzia były darmowe, albo
uprzednio zakupione przez Nordeę. Czas wykorzystany na rozwój
automatyzacji był i tak czasem \quotation{straconym}. Tj. zamiast
oczekiwania na coś, zabieraliśmy się za usprawnienia.

\subsection[zrób-to-sam]{Zrób to sam!}

W każdym zespole można zrobić to co my. Rozwój odpowiednio
zaprojektowanych narzędzi zawsze przyniesie oszczędności w długim
okresie czasu. Mimo że my użyliśmy części płatnych narzędzi, wcale nie
jest to konieczne.

\section[bezkosztowe-narzędzia]{Bezkosztowe narzędzia}

W zasadzie każde platne narzędzie ma swój darmowy odpowiednik. Bitbucket
- GitLab, Bamboo - TeamCity itd. Do tego dochodzi cała masa narzedzi po
prostu darmowych, takich jak Argo, Ansible czy zwykły Python.
Wykorzystanie tych możliwości pozwala na uzyskanie zysków bez żadnego
wkładu finansowego. Nie jest również żadnym problemem, a jest prostym
sposobem na uzyskanie pochwał to, że można użyć ich do analityki (np.
Grafana), by badać co się właściwie dzieje. Można również traktować
istniejące narzędzia w firmie jako \quotation{darmowe} i na podstawie
ich rozszerzonych często możliwości, a przynajmniej przy wykorzystaniu
wsparcia technicznego, rozwijać ekosystem narzędziowy w wymagane strony.

\section[grywalizacje]{Grywalizacje}

Zachętą do regularnego kontrybuowania do wspólnych narzędzi może być
wykorzystanie technik grywalizacyjnych, które u nas skończyły się
również implementacją całego systemu do obsługi \quotation{gry}, co
znowu było nie dość, że rozwinięciem obecnych narzędzi, to jeszcze
niejako nagrodą za uczestnictwo (zamiast kolejnych testów big data, mały
sklepik internetowy).

\section[to-się-opłaca-każdemu]{To się opłaca każdemu}

Najważniejsze w tym wszystkim jest to, że każdej firmie opłaca się
stworzyć własny ekosystem przy użyciu darmowych narzędzi. Wprowadzenie
CI/CD jest często oczywiste, ale nie zawsze oczywistym jest to, że w te
mechanizmy można i należy wpleść odpowiedzialności testerskie. W ten
sposób, obok automatycznego wykonywania testów programisty i
raportowania nowych wersji, nagle okazuje się, że kosztem kilku dni poza
normalnymi obowiązkami pracowniczymi, można zautomatyzować wszystko. A
potem skupić sie na tej fajnej robocie.

\subsection[źródła]{Źródła}

\startitemize[n,packed][stopper=.]
\item
  Python, \useURL[url1][https://www.python.org]\from[url1]
\item
  Jenkins, \useURL[url2][https://jenkins.io]\from[url2]
\item
  Argo,
  \useURL[url3][https://blog.argoproj.io/tagged/workflow-automation]\from[url3]
\item
  Ansible, \useURL[url4][https://www.ansible.com]\from[url4]
\item
  Narzędzia Atlassian,
  \useURL[url5][https://www.atlassian.com]\from[url5]
\item
  Cenniki GitLab,
  \useURL[url6][https://about.gitlab.com/pricing]\from[url6]
\stopitemize


\stoptext
