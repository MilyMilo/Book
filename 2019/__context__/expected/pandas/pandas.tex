\usemodule[pycon-yyyy]
\starttext

\Title{How Pythons and Pandas can solve real-life problems when working with Big Data}
\Author{Łukasz Szweda}
\MakeTitlePage

\subsection[introduction]{Introduction}

This article aims to provide necessary background for the presentation.
Many concepts will be repeated in both media, but having read the
article, the viewer will gain a better understanding of why the actions
performed were needed and how they were performed.

\subsection[the-problem]{The Problem}

Nordea Bank encountered some troubles, that would later be viewed as the
biggest credit risk project in Europe\ldots{} at least so would they
say. The vast amount of data processed for the purposes of the
aforementioned credit risk calculation proved to be inadequately
handled, Nordea risked fines and reputation loss. This is why the MDP
project was initialized.

\section[big-data]{Big Data}

By very definition the project stemmed from the fact that the bank did
not understand handling of so called Big Data, which for the purposes of
this topic should be defined as data which is mostly unstructured,
varies in sources and formats and it is critically important to handle
it securely and precisely.

\section[quick-deadline]{Quick Deadline}

Approaching deadlines meant that there was no time for fancy tools and
processes. The data needed to get ingested into Apache Hadoop data
platform sooner rather than later. Many workarounds were designed, such
as treating everything that is not easily definable as string. The data
was later supposed to be repurposed/transformed, so this did not matter
in the given moment.

\section[failing-tools]{Failing Tools}

We have hoped to reuse the ingest framework that was created in Nordea.
However none of the patterns/generic methods fit. The data was coming in
as fixed width files, CSVs, excels, while the framework recognized
mostly mainframe format and data coming from databases. The team said
they can be ready by the end of the year. This was a no-go - we needed
the data inside the systems in matter of weeks, not months.

\subsection[the-solution]{The Solution}

The solution was to sic loose an Expert IT Developer to do it. The
Chosen One was me :-). I cracked my knuckles and began to work on
ingesting the data.

\section[python]{Python}

Lo' and behold, Python 3 seemed up to the task. Armed with just Jupyter
Notebook, Numpy and Pandas, this language, allowing easy prototyping and
fast programming showed its potential in its entirety when parsing the
various data sources given to us by the business.

\section[apache-hadoop]{Apache Hadoop}

Apache Hadoop is an ecosystem of tools allowing handling Big Data. It
consists of many software solutions, ranging from schedulers, file
systems to metadata bases and query engines.

\section[aws-dask]{AWS, Dask}

Much earlier there were some plans and POCs regarding usage of various
tools improving the performance of the solution. At first we wanted to
run it against our Hadoop Cluster - using Dask - but it turned out the
cluster availability actually hindered the processes instead of
empowering it. We have also checked possibilities of AWS, but then it
turned out that cloud services for critical data is a no-go at Nordea
Bank.

\section[pandas]{Pandas}

Turns out that Pandas itself is more then sufficient given a clear
purpose and reusable target. It is a well-known library designed for
handling data models and structures. They are kept in-memory as a unique
object called the Dataframe, which is best described as a minimum viable
table. Fortunately, most of the needed formats were already handled by
the library, so the only thing needed was in many cases the optimization
of input and corner-handling.

\section[jupyter-notebook]{Jupyter Notebook}

What was very helpful is the usage of Jupyter Notebook, which will be
featured extensively in the presentation. The fast prototyping allowed
by Python is taken to another level with this persistence-keeping tool.
Jupyter Notebook was designed and developed for use by data scientists.
Its main draw is the formatting of code, which accepts normal code and
keeps it in \quotation{cells}. These cells are persistent, and the
objects defined in one cell can be reused in another. This allows for
decomposition of methods and functions without having to handle data
storage, so that instead of storing the parsed data on hard drive (slow)
it can be reused easily from memory (maximum fast!).

\subsection[lessons-learnt]{Lessons Learnt}

\startitemize[packed]
\item
  you don't need complex systems to handle even terabytes of data
\item
  Python is perfectly fast for most usages
  \startitemize[packed]
  \item
    When Python itself is too slow, Numpy/Cython is fast enough
  \stopitemize
\item
  One developer with great freedom and well-defined requirement can do
  the work of a big team with chaotically managed backlog
\stopitemize

\subsection[sources]{Sources}

\startitemize[n,packed][stopper=.]
\item
  Python, \useURL[url1][http://www.python.org/]\from[url1]
\item
  Jupyter Notebook, \useURL[url2][https://jupyter.org/]\from[url2]
\item
  Pandas, \useURL[url3][https://pandas.pydata.org/]\from[url3]
\item
  Dask, \useURL[url4][https://dask.org/]\from[url4]
\item
  Big Data,
  \useURL[url5][https://en.wikipedia.org/wiki/Big_data]\from[url5]
\item
  Apache Hadoop, \useURL[url6][https://hadoop.apache.org/]\from[url6]
\item
  Amazon Web Services, \useURL[url7][https://aws.amazon.com/]\from[url7]
\stopitemize


\stoptext
