\usemodule[pycon-yyyy]
\starttext

\Title{Jak zacząć udzielać się w projektach Open Source?}
\Author{Piotr Gaczkowski}
\MakeTitlePage

\subsection[dlaczego-warto-zacząć]{Dlaczego warto zacząć?}

Dlaczego warto udzielać się w projektach Open Source? Dlaczego warto
spędzać czas na pisaniu kodu lub tłumaczeniu zamiast iść na spacer? Jest
kilka powodów. Wszystkie wynikają z faktu, że wkład w Open Source jest
publicznie widoczny. Każdy może zobaczyć, co robią wszyscy inni. I to
jest dla nas wielką zaletą!

\subsection[zwiększasz-swoje-kompetencje]{Zwiększasz swoje kompetencje}

Gdy piszesz fragment kodu, projektujesz interfejs lub tworzysz grafikę,
uczysz się. To świetne ćwiczenie dla osób poszukujących zmiany kariery.
W Open Source nikt nie dba o to, czy pracujesz jako starszy programista,
czy jako ogrodnik. Liczy się jakość Twojego wkładu. A kiedy dowiedziesz
swojej wartości w projekcie Open Source, zgromadzisz do tego czasu sporo
cennego doświadczenia. Nie będziesz czuć się oszustem, lub początkującą
osobą świeżo po kursie. Będziesz szanowanym członkiem społeczności s
Twoje doknania każdy będzie mógł sprawdzić.

Działa to nawet wtedy, gdy utkniesz w pracy z jedną technologią
wyobrażając sobie swoją przyszłość z czymś zupełnie innym. Powiedzmy, że
jesteś programistą COBOL marzącym o znalezieniu pracy w UX. Jednym ze
sposóbów by to osiągnąć jest zacząć ulepszać UX w wybranym projekcie
Open Source i wysłanie swoich uwag opiekunom projektu. Po ich
zaakceptowaniu możesz dodać nową pozycję do swojego CV: projektant UX
dla projektu Open Source. W ten sposób ucząć się czegoś przydatnego
jednocześnie budujesz portfolio.

\subsection[poprawiasz-swoją-widoczność]{Poprawiasz swoją widoczność}

Dołączając do społeczności, takiej jak projekt Open Source, natychmiast
zyskujesz nowych znajomych. Spotykasz innych ludzi którzy również biorą
udział w projekcie. Możesz uczyć się od nich i wymieniać z nimi pomysły.

Drugą korzyścią z dołączenia do takiej społeczności jest to, że
wszystkie inne Twoje kanały komunikacji automatycznie stają się bardziej
widoczne. Na przykład użytkownicy projektu Open Source, w którym
uczestniczysz, mogą dowiedzieć się o Twoim kanale YouTube lub znaleźć
Cię na LinkedIn, aby zaoferować Ci pracę.

To prowadzi nas do następnego punktu.

Obecnie rekruterzy IT często przeglądają GitHub w poszukiwaniu nowych
talentów. Upubliczniając kod, tworzysz życiorys, który każdy może
zweryfikować. Rekruterzy nie muszą kontaktować się z Twoimi byłymi
pracodawcami lub klientami. A jeśli Twój profil GitHub prowadzi do
Twojego profilu LinkedIn lub strony głównej zwiększasz swoje szanse na
znalezienie wymarzonej pracy.

Pamiętaj, że nie musisz nawet mieć doświadczenia komercyjnego w tej
konkretnej dziedzinie. W przypadku osób rekrutujących, Twój wkład w Open
Source może okazać się równie ważny co poprzednie miejsca i stanowiska
pracy!

\subsection[znajdujesz-zatrudnienie-lub-nowych-klientów]{Znajdujesz
zatrudnienie lub nowych klientów}

Jeśli dobrze wykonujesz swoją robotę, możesz otrzymać płatną ofertę
pracy nad projektem Open Source. Ta możliwość dotyczy głównie tych
projektów, które mają stabilne wsparcie finansowe ze strony korporacji
lub organizacji non-profit.

\subsection[znajdujesz-nowych-pracowników]{Znajdujesz nowych
pracowników}

Działa to również w drugą stronę -- możesz znaleźć potencjalnych
pracowników. Jeśli robisz coś fajnego i przydatnego dla innych,
opublikuj to jako projekt Open Source! Istnieje szansa, że ktoś będzie
chciał ulepszyć Twój projekt lub pracować dla Ciebie.

\subsection[robisz-coś-dla-innych]{Robisz coś dla innych}

Kolejną zaletą bycia częścią tych projektów jest możliwość zrobienia
czegoś dla innych i zdobycia ich wdzięczności. Wszystkie otrzymane listy
z podziękowaniami sprawią, że Twój poświecony czas okaże się świetną
inwestycją!

\subsection[masz-dostęp-do-ładnych-statystyk]{Masz dostęp do ładnych
statystyk}

Czy jesteś fanem grywalizacji? Niektórzy naprawdę lubią statystyki i
właśnie dla nich platformy takie jak GitHub oferują coś wyjątkowego.
Każda czynność, którą podejmujemy w GitHub - największej społeczności
Open Source - jest oznaczona zielonym kwadratem w wielkim kalendarzu. Im
więcej rzeczy robimy danego dnia, tym ciemniejszy jest kwadrat. Gdy
będziemy udzielać się przez kilka kolejnych dni, GitHub policzy go jako
serię i zapisze ten \quotation{życiowy rekord} na przyszłość.

To motywuje do pobicia najlepszych, jak dotąd, wyników.

\subsection[jak-znaleźć-swój-pierwszy-projekt]{Jak znaleźć swój pierwszy
projekt?}

Istnieje kilka sposobów. Przede wszystkim duzi gracze, tacy jak Firefox
czy Android, mają dużą bazę użytkowników i potrzebują wielu zmian i
ulepszeń. Po wprowadzeniu zmiany w takim projekcie stanie się ona
widoczna dla milionów. Możesz odwiedzić GitHub i zobaczyć, które
projekty są najpopularniejsze. Na przykład, w momencie pisania tego
artykułu repozytorium najczęsciej oznaczonym gwiazdką (co świadczy o
popularności) jest
\useURL[url1][https://github.com/freeCodeCamp/freeCodeCamp][][freeCodeCamp]\from[url1].

Niekoniecznie poleciłbym to pierwsze podejście, jeśli jesteś
nowicjuszem. Główna zaleta wielkich projektów jest także ich
przekleństwem. Każdy chce wnieść swój wkład, ale jest ograniczona liczba
opiekunów, którzy są w stanie odpowiedzieć na Twoje zgłoszenia. Jeśli
naprawdę chcesz wywrzeć wpływ, możesz zamiast tego wypróbować nieco
mniejsze projekty. \useURL[url2][https://github.com/trending][][GitHub
Trending]\from[url2] to miejsce, w którym można znaleźć takie przypadki.

Innym podejściem jest wybranie projektu typu \quotation{awesome} za
pierwszy cel. Nazwijmy to podejściem efektywnym energetycznie. Projekty
typu \quotation{awesome} to skompilowane i wyselekcjonowane listy
odnośników. Na przykład
\useURL[url3][https://github.com/alebcay/awesome-shell][][awesome-shell]\from[url3]
zbiera interesujące projekty CLI, a
\useURL[url4][https://github.com/sorrycc/awesome-javascript][][awesome-javascript]\from[url4]
zajmuje się wszystkimi nowymi lśniącymi frameworkami, które pojawiają
się każdego dnia. Jest też lista
\useURL[url5][https://github.com/hackerkid/Mind-Expanding-Books][][książek
rozwijających umysł]\from[url5] oraz, oczywiście,
\useURL[url6][https://github.com/sindresorhus/awesome][][lista
{\em list} typu \quotation{awesome}]\from[url6].

Dlaczego to jest dobre podejście? Wysiłek wymagany do faktycznego
wniesienia czegoś pożytecznego jest dość niski. W wyniku tego istnieje
duża szansa na zwiększenie dawki dopaminy w związku z dobrze wykonaną
pracy. To zwiększenie dopaminy powinno uruchomić chęć dalszego
udzielania się i tak dalej. Gdy piłkę raz wprawimy w ruch, trudno będzie
ją zatrzymać. Tak właśnie tworzą się nawyki.

Ostatnim podejściem, które polecam, jest dołączenie do jednej z imprez
Open Source. Jedną z nich jest obchodzony każdego roku w październiku
Hacktoberfest. Celem jest w ciągu miesiąca dokonać pięciu propozycji
zmian w dowolnych projektach na GitHub. Każdy, kto wypełni to zadanie,
otrzymuje od sponsorów fanty (takie jak koszulki i naklejki). Aby
ułatwić uczestnikom rozpoczęcie zabawy, niektóre przykładowe projekty są
wymienione na stronie internetowej.

Jeśli chcesz wspierać Open Source w sposób regularny, znajdź coś, z
czego korzystasz na co dzień. W ten sposób Twoja praca przyniesie
korzyści zarówno Tobie jak i społeczności. Jest to zdecydowanie
najlepszy sposób na zapewnienie wysokiej jakości pracy!


\stoptext
