\usemodule[pycon-yyyy]
\starttext

\Title{Bluetooth Low Energy z PyBluez}
\Author{Marcin Bardź}
\Author{Krzysztof Czarnota}
\MakeTitlePage

\subsection[wprowadzenie]{Wprowadzenie}

Bluetooth Low Energy (BLE) został wprowadzony jako część specyfikacji
Bluetooth 4.0. Technologia ta umożliwia dostęp do zewnętrznych urządzeń
praktycznie z każdego systemu mobilnego. Jest to technologia połączenia
bezprzewodowego o niskiej przepustowości, bardzo efektywna pod względem
poboru mocy i kosztów implementacji. Dzięki wyposażeniu urządzeń
mobilnych w układy obsługujące technologię BLE oraz standaryzację
komunikacji przez wykorzystanie Generic Attribute Profile (GATT),
obserwujemy wysoki poziom adaptacji tej technologii na rynku urządzeń
internetu rzeczy (IoT).

\subsection[standard-bluetooth-low-energy]{Standard Bluetooth Low
Energy}

Bluetooth Low Energy został wprowadzony przez Bluetooth Special Interest
Group (SIG) w czerwcu 2010 roku wraz z wersją 4.0 specyfikacji
Bluetooth. Specyfikacja Bluetooth obejmuje zarówno klasyczny bluetooth
(znany standard bezprzewodowy, który jest powszechnie stosowany w
urządzeniach konsumenckich od wielu lat), jak i Bluetooth Low Energy
(nowy, wysoce zoptymalizowany standard). Te dwa standardy komunikacji
bezprzewodowej różnią się właściwie na każdej warstwie i nie są ze sobą
bezpośrednio kompatybilne.

\subsection[komunikacja-w-standardzie-ble]{Komunikacja w standardzie
BLE}

Urządzenie BLE może komunikować się w trybie rozgłoszeniowym lub
połączeniowym. Tryb rozgłoszeniowy nie wymaga połączenia i pozwala na
jednokierunkowe wysyłanie danych do dowolnego urządzenia skanującego
znajdującego się w zasięgu. Jest to jedyna możliwość przesłania danych
do więcej niż jednego urządzenia w tym samym czasie. Standardowy pakiet
rozgłoszeniowy zawiera 31 bajtów, które opisują urządzenie oraz jego
możliwości. Pakiet rozgłoszeniowy może również zawierać dowolne
niestandardowe dane. Jeżeli rozgłaszane dane są dłuższe niż 31 bajtów,
BLE umożliwia przesłanie drugiego pakietu danych (Scan Response) o tej
samej długości, w odpowiedzi na żądanie urządzenia skanującego. Główną
wadą komunikacji w trybie rozgłoszeniowym jest brak jakichkolwiek
mechanizmów zabezpieczających komunikację \footnote{Robert Davidson,
  Akiba, Carles Cufí, Kevin Townsend - Getting Started with Bluetooth
  Low Energy.
  https://www.oreilly.com/library/view/getting-started-with/9781491900550/}.

W przypadku, gdy wymagana jest transmisja danych o długości większej niż
62 bajty lub komunikacja dwukierunkowa, urządzenia muszą nawiązać
połączenie. Połączenie BLE zapewnia trwały tunel transmisyjny pomiędzy
dwoma równorzędnymi urządzeniami, co czyni komunikację z natury
prywatną. Aby zainicjować połączenie, urządzenie centralne (master)
wyszukuje pakiet rozgłoszeniowy urządzenia peryferyjnego (slave) i
wysyła żądanie nawiązania połączenia. Należy zauważyć, że role
przypisane urządzeniom podczas nawiązywania połączenia nie mają wpływu
na późniejszą komunikację i dane mogą być wysyłane niezależnie w obu
kierunkach. Połączenia pozwalają na znacznie bardziej złożony model
danych, a dzięki temu na lepszą organizację komunikacji, poprzez
zastosowanie dodatkowych warstw protokołu, a dokładnie Generic Attribute
Profile (GATT) \footnote{Robert Davidson, Akiba, Carles Cufí, Kevin
  Townsend - Getting Started with Bluetooth Low Energy.
  https://www.oreilly.com/library/view/getting-started-with/9781491900550/}.

Podczas nawiązywania połączenia, urządzenie peryferyjne zasugeruje
\quotation{interwał połączenia} z urządzeniem centralnym. Jest to czas
pomiędzy dwoma zdarzeniami transmisji danych (zdarzeniami BLE) w trakcie
komunikacji urządzenia centralnego i peryferyjnego. Wartość teoretyczna
waha się od 7,5 ms do 4 s (z przyrostem 1,25 ms). Ostateczna wartość
tego parametru zdeterminowana jest przez urządzenie nadrzędne, jednak
należy wyważyć wymaganą przepustowość oraz zużycie energii.

\subsection[generic-attribute-profile-gatt]{Generic Attribute Profile
(GATT)}

Generic Attribute Profile definiuje sposób, w jaki dwa urządzenia
Bluetooth Low Energy przekazują dane między sobą za pomocą koncepcji
zwanych usługami (Services) i cechami (Characteristics). Lista usług
wraz z ich cechami składa się na profil GATT urządzenia. GATT
wykorzystuje jako warstwę transportową ogólny protokół danych BLE zwany
Attribute Protocol (ATT), który jest używany do przechowywania usług,
cech oraz powiązanych danych w prostej tabeli z wykorzystaniem
identyfikatorów (UUID) \footnote{https://learn.adafruit.com/introduction-to-bluetooth-low-energy/gatt}.
Specyfikacja GATT definiuje również zalecenia dla wszystkich profili
opartych na GATT (zaakeceptowane przez Bluetooth SIG), które obejmują
precyzyjne przypadki użycia konkretnych profili, usług i cech, a dzięki
temu zapewniają interoperacyjność pomiędzy urządzeniami pochodzącymi od
różnych producentów. Wszystkie standardowe profile BLE są zatem oparte
na GATT i muszą być z nim zgodne, aby działały prawidłowo. To czyni z
GATT kluczową sekcję specyfikacji BLE, ponieważ każdy pojedynczy element
danych istotny dla zastosowań lub użytkowników musi być sformatowany,
opakowany i wysłany zgodnie z jego zasadami \footnote{Robert Davidson,
  Akiba, Carles Cufí, Kevin Townsend - Getting Started with Bluetooth
  Low Energy.
  https://www.oreilly.com/library/view/getting-started-with/9781491900550/}.

\subsection[transakcje-gatt]{Transakcje GATT}

Urządzenie peryferyjne (GATT Sever) definiuje listę usług w oparciu o
wykorzystany profil lub wymagania projektowe. Pełny wykaz oficjalnie
przyjętych profili opartych na GATT jest dostępny na Bluetooth Developer
Portal \footnote{https://www.bluetooth.com/specifications/gatt/}. Usługi
zdefiniowane przez profil porządkują dane w logiczne jednostki
zawierające jedną lub więcej cech. Każda usługa posiada własny UUID,
który może być 16 bitowy (w przypadku oficjalnie przyjętej usługi) lub
128 bitowy (dla niestandardowych usług). Lista oficjalnie przyjętych
usług wraz z ich identyfikatorami również jest dostępna na Bluetooth
Developer Portal \footnote{https://www.bluetooth.com/specifications/gatt/services/}.
Cecha, to pojedynczy punkt danych, który może w zależności od potrzeb
zawierać dane proste lub złożone. Długość danych ograniczona jest do 512
bajtów przez specyfikację. Podobnie jak usługi, cechy posiadają swoje 16
lub 128 bitowe UUID i podobnie jak w poprzednich przypadkach Bluetooth
Developer Portal podaje listę oficjalnie przyjętych cech \footnote{https://www.bluetooth.com/specifications/gatt/characteristics/}.
Każda transakcja jest inicjowana przez urządzenie master (GATT Client),
które może zażądać odczytu lub zapisu bufora danych cechy. Dostęp do
cech może być ograniczony przez uprawnienia ATT, na które składają się:

\startitemize[packed]
\item
  uprawnienia dostępu (none, readable, writable, readable and writable),
\item
  uprawnienia szyfrowania (no encryption required, unauthenticated
  encryption required, authenticated encryption required),
\item
  uprawnienia autoryzacji (no authorization required, authorization
  required) \footnote{Robert Davidson, Akiba, Carles Cufí, Kevin
    Townsend - Getting Started with Bluetooth Low Energy.
    https://www.oreilly.com/library/view/getting-started-with/9781491900550/}.
\stopitemize

Esencją komunikacji w standardzie BLE jest manipulacja wartości
zidentyfikowanych dzięki cechom zdefinowanym przez profil Generic
Attribute Profile urządzenia. Wymiana danych z urządzeniami tego samego
typu jest ustandaryzowana dzięki wykorzystaniu profili GATT
zaakceptowanych przez SIG, co pozwala na nawiązanie komunikacji z
urządzeniem do którego nie posiadmy specyfikacji i daje szerokie pole do
eksploracji urządzeń BLE.


\stoptext
