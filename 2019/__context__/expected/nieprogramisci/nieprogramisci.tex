\usemodule[pycon-yyyy]
\starttext


\Title{Jak pracować z programującymi nieprogramistami}
\Author{Jacek Bzdak}
\Author{Magdalena Wójcik}
\MakeTitlePage

\subsection[streszczenie]{Streszczenie}

W prezentacji chcemy przedstawić trudności jakie pojawiają się w pracy
na linii programiści - programujący nieprogramiści, które wynikają z
naszego doświadczenia. Dotyczą one zarówno kwestii jakości kodu,
wykorzystywanych narzędzi czy środowiska pracy. Dla każdego z problemów
zaprezentujemy rozwiązania, które stosujemy w naszej codziennej pracy.
Prezentacja będzie opierała się na dwóch punktach widzenia - programisty
i Data Scientistki. Poniższy artykuł zawiera spis technicznym zagadnień.
W artykule poruszamy głównie kwestie związane z zespołem Data Science,
ale w prezentacji powiemy też współpracy z innymi grupami, np.
elektronikami.

\subsection[wstęp]{Wstęp}

\section[disclaimer]{Disclaimer}

\startitemize[n,packed][stopper=.]
\item
  Podane niżej pomysły sprawdzają się w \quotation{młodym} zespole w
  consultingowym startupie, który realizuje dużo małych projektów
  \footnote{\useURL[url3][https://logicai.io/]\from[url3]},
\item
  Relatywnie rzadko robimy projekty Big Data, większość z nich da się
  zrobić na jednej maszynie (nawet jeśli jest to \quotation{duża}
  instancja),
\item
  Jeśli masz terabajty danych i od lat tworzysz produkt możliwe, że
  rozwiązania te mogą nie być optymalne.
\stopitemize

\section[jak-wygląda-praca-data-scientista]{Jak wygląda praca Data
Scientista?}

Projekty Data Science (DS) są podobne do projektów stricte
programistycznych, ale mają też swoje specyficzne elementy, które
wymienimy poniżej.

Zwykle praca przebiega następująco:

\startitemize[n,packed][stopper=.]
\item
  Rozmowa z klientem, zebranie wymagań i potrzeb, określenie jaki cel
  biznesowy ma osiągnąć model Machine Learningowy (ML);
\item
  Analiza i czyszczenie danych, co może zająć nawet 90\letterpercent{}
  czasu całego projektu, jeśli dane zawierają dużo błędów i szumu; W
  ramach tego kroku powstają tzw. Transformery czyli skrypty
  przygotowujące dane do postaci odpowiedniej dla modelu ML;
\item
  Trenowanie modelu ML;
\item
  Przekazanie wyników do klienta. Wynik może być zarówno gotowym
  produktem, mikroserwisem Data Science, jak i raportem/prezentacją.
\stopitemize

\subsection[praca-z-data-scientistamikami]{Praca z Data
Scientistami\letterbackslash{}kami}

\section[wspólne-ustalenie-api-które-jest-zrozumiałe-dla-obu-stron]{Wspólne
ustalenie API, które jest zrozumiałe dla obu stron}

Często da się opracować API między częścią {\em programistyczną} oraz
częścią {\em Data Sciencową}.

Ważne uwagi:

\startitemize[n,packed][stopper=.]
\item
  API musi być zrozumiałe dla wszystkich, w praktyce podlega bardzo
  dokładnemu przeglądowi kodu, i {\bf wszyscy} muszą być pewni, że jest
  dobre,
\item
  Należy być otwartym na jego refaktoryzację.
\stopitemize

\subsection[co-może-się-nie-udać]{Co może się nie udać?}

Narzucenie API zrobionego tak, żeby było \quotation{elegancko} i zgodnie
z obowiązującą modą programistyczną, bez patrzenia na przejrzystość API.
Dla nieprogramistycznej części zespołu to przepis na ciągłe ignorowanie
API i świadczenie supportu programistycznego przy nawet prostych
zmianach.

\section[przeglądy-kodu-jako-okazja-do-ulepszenia-praktyk]{Przeglądy
kodu jako okazja do ulepszenia praktyk}

Czasem przeglądy kodu są nieprzyjemne, ale w naszej firmie dbamy o to,
żeby było miło, więc przeglądy kodu {\em nie są} okazją do {\em gate
keepingu} (czyli nie służą do tego, żeby odrzucać zmiany, które nie
spełniają standardów). Przeglądy są dobrą okazją do poprawy praktyk
przyjętych w firmie.

W większości projektów mamy ciągłą integrację i testy, co pozwala
zarządzać ryzykiem wprowadzenia regresji na produkcję.

Nasza procedura przeglądania kodu:

\startitemize[packed]
\item
  Znajdź kilka rzeczy, których poprawa zmieni najwięcej;
\item
  Przekonaj osobę zgłaszającą kod do tego, że proponowane poprawki
  pomogą, albo: daj się przekonać, że nie ma sensu tego robić, bo nie
  będzie istotnej zmiany;
\item
  Potraktujcie poprawę kodu jako wspólną odpowiedzialność.
\stopitemize

Przeglądamy cały kod trafiający do repozytorium, łącznie z notebookami
Jupytera i prototypowym kodem. Dbamy, żeby produkcyjny kod Pythona miał
zdecydowanie wyższą wynikową jakość, niż części prototypowane.

\section[brak-silo-mentality]{Brak \quotation{silo mentality}}

Mentalność silosu \footnote{\useURL[url3][https://zapier.com/blog/organizational-silos/]\from[url3]}
to sytuacja, w której dwa zespoły nie dzielą się informacjami i
współpracują tylko na podstawie formalnych procedur.

Sytuacja, w której nad projektem pracuje jeden team zawierający i DS, i
programistki\letterbackslash{}-ów jest dobra, ponieważ wymusza to naukę
z obu stron.

\subsection[wdrożenia]{Wdrożenia}

Do wdrożeń używamy dockera \footnote{\useURL[url4][https://www.docker.com/]\from[url4]},
pozwala to bezpiecznie \quotation{zapakować} wszystkie zależności oraz
wytrenowany model (kolejne wersje modelu wydawane są jako kolejne wersje
obrazu).

Taka metoda pozwala też na bardzo łatwe wdrożenie u klienta.

Zależności projektu instalujemy standardowo za pomocą polecenia
\type{pip}. Generujemy pliki zawierające dokładnie wersje wszystkich
zależności (również zależności tranzytywnych). Używamy do tego narzędzi
\type{pip-tools} \footnote{\useURL[url5][\%5Bhttps://github.com/jazzband/pip-tools][][https://github.com/jazzband/pip-tools]\from[url5]}.
Pip-tools zawiera polecenie \type{pip-compile}, które generuje plik z
zamrożonymi zależnościami, tak wygenerowany plik można zainstalować za
pomocą \type{pip install -r}. Dzięki temu podczas instalacji projektu
nie ma potrzeby używania niestandardowych narzędzi (co upraszcza
istotnie cały proces).

Do wdrożeń nie używamy Anacondy \footnote{\useURL[url6][https://www.anaconda.com/]\from[url6]}.
Anaconda to manager pakietów, który instaluje typowe paczki Data
Science, wraz z ich wszystkimi skompilowanymi zależnościami (np.
bibliotekami algebraicznymi). Jest czasem wygodny i bardzo lubiany przez
Data Scientistki\letterbackslash{}-ów, ale nie używamy go we
wdrożeniach, ponieważ \quotation{przypinanie} zależności jest w nim mało
praktyczne.

Świetnie działa Continuous Deploymen. Jest on szczególnie istotny w
teamie w którym nie każdy ma chęć i umiejętności robienia wdrożeń.

\subsection[jupyter-notebook]{Jupyter notebook}

Jupyter \footnote{\useURL[url7][https://jupyter.org/]\from[url7]} jest
interaktywnym webowym środowiskiem, w którym można pisać kod i od razu
obserwować jego wyniki. Jeśli go nie znasz, zainstaluj i zacznij używać
do prototypowania (nie tylko w Data Science!).

Świetnie się w nim pisze prototypowy kod Data Science ponieważ:

\startitemize[packed]
\item
  Blisko kodu można umieścić wynik jego działania (wykresy, statystyki
  itp.);
\item
  Niektóre kroki obliczeniowe mogą trwać godzinami (np. trenowanie
  modelu) a Jupyter cechuje ich wyniki;
\item
  Niektóre notebooki liczą godzinami, więc wygodnie jest mieć gotowy
  zrenderowany raport, który można pokazać innym (wyniki i wykresy są w
  pliku z notebookiem).
\stopitemize

Jupytera można użyć też do bardzo wygodnej i efektywnej pracy na
zdalnych maszynach (patrz: \quotation{Praca na instancjach}).

Dlaczego notebooki są czasem mało fajne:

\startitemize[packed]
\item
  Pliki, których używa Jupyter do przechowywania notebooków nie są
  czytelne dla ludzi. Technicznie są plikami JSON zawierającymi kod
  każdej komórki i wynikami zrealizowanych obliczeń. Ten format danych
  nie nadaje się do tworzenia diff/PR;
\item
  Nie można ich wdrożyć na produkcję (sprawdzić czy nie Netflix)
  \footnote{\useURL[url8][https://medium.com/netflix-techblog/notebook-innovation-591ee3221233]\from[url8]};
\item
  Wspierają pisanie ad-hocowego throw-away code (patrz:
  \quotation{Throw-away code});
\stopitemize

\section[przeglądy-kodu-a-jupyter]{Przeglądy kodu a Jupyter}

\startitemize[packed]
\item
  Używamy wtyczki \type{jupytext} \footnote{\useURL[url9][https://github.com/mwouts/jupytext]\from[url9]},
  która do plików notebooka (\type{*.ipynb}) Dodaje siostrzane pliki
  \type{*.py} (które potem można otworzyć w dowolnym IDE oraz
  Jupyterze);
\item
  Pliki notatników (\type{*.ipynb}) nie trafiają do repozytorium, chyba
  że zawierają wartościowe wyniki (tj. pliki \type{*.ipynb} ignorowane i
  trzeba je ręcznie dodać);
\item
  Commitowane notebooki podlegają normalnemu przeglądowi kodu w postaci
  \type{*.py} (aczkolwiek rozumiemy, że nie wszystkie wymogi jakości
  mają w nich sens);
\stopitemize

\subsection[biblioteka-pandas]{Biblioteka Pandas}

Pandas \footnote{\useURL[url10][https://pandas.pydata.org/]\from[url10]}
jest często używane z Jupyterem, pozwala na eksploracyjne przeglądanie
danych, wykonywanie przekształceń i agregacji, podobnie jak w SQL.

\section[problemy]{Problemy}

Główną wadą tej biblioteki jest trzymanie wszystkiego w pamięci RAM co,
po prostu, nie zadziała dla dużych zbiorów danych. Często dobrym
rozwiązaniem jest wrzucenie skryptu z Pandasem na dużą instancję, bo
koszt maszyny jest szybko przewyższony przez oszczędności wynikające ze
zwiększonej efektywności programistki/ty.

Pandas ma też czasem nieintuicyjną charakterystykę wydajnościową -
rozwiązaniem jest testowanie wydajności każdego Transformera niezależnie
i przepisywanie kodu będącego wąskim gardłem.

\subsection[testowanie-kodu]{Testowanie kodu}

Automatyczne testowanie kodu jest bardzo przydatnym narzędziem w
projektach DS.

Warto testować:

\startitemize[packed]
\item
  Każdy Transformer danych, ponieważ ew. błędy (np. wycieki danych)
  najczęściej znajdują się tutaj, a nie w modelu, a błędy wprowadzone
  podczas obróbki danych są bardzo trudne do znalezienia.
\item
  Warto zrobić \quotation{smoke testy} \footnote{\useURL[url11][https://en.wikipedia.org/wiki/Smoke_testing_(software)]\from[url11]}
  całego procesu przetwarzania danych, czyli: uruchamiamy proces z
  danymi wejściowymi w dobrym formacie i patrzymy czy wychodzą dane w
  dobrym formacie.
\stopitemize

Automatyczne testowanie samych modeli ma bardzo ograniczony sens,
przestrzeń wyników generowanych przez model jest bardzo duża i nie da
się przetestować wszystkich warunków brzegowych. Zapewnienie poprawnego
działania modelu jest odpowiedzialnością osoby, która go przygotowuje.

Poniżej typowe problemy, które można napotkać przy budowaniu modeli,
które są trudne do automatycznego testowania:

\startitemize[n,packed][stopper=.]
\item
  Model może mieć bardzo wysoką dokładność ze względu na wyciek w
  danych. Na przykład może bardzo dobrze zdiagnozować raka na podstawie
  tego, że pacjent w historii choroby ma długi pobyt w szpitalu
  onkologicznym;
\item
  Model może działać poprawnie, ale mieć uprzedzenia, np. rasowe, albo
  do płci.
\stopitemize

\subsection[praca-na-instancjach]{Praca na instancjach}

Algorytmy DS zużywają dużo pamięci RAM (czasem setki GB), a czasem
potrzebują kart graficznych. Nie zawsze da się je uruchomić lokalnie, a
szczególnie z Zadowalającą wydajnością.

\section[problemy-1]{Problemy}

Dużym problemem jest konfiguracja zdalnej instancji, osoby z Data
Science nie zawsze czują się komfortowo z Linuxem i często konfiguracja,
która intuicyjnie powinna być szybka, zajmuje np. cały dzień.

Najprawdopodobniej rozwiążemy to poprzez standaryzację szablonu
projektu, tak by postawienie instancji zawsze wymagało pobrania
repozytorium, pobrania danych i napisania \type{docker-compose up}.

\section[rozwiązanie-modelowe]{Rozwiązanie modelowe}

Jak najwięcej należy zrobić bez zdalnej maszyny. W praktyce
90\letterpercent{} problemów da się wychwycić sprawdzając model na małej
próbce danych. W praktyce cały kod Pythona powstaje lokalnie.

W takim wypadku cały model sprawdzany jest na instancji na pełnych
danych, czasem mamy automatyczne skrypty, które tworzą instancję,
pobierają dane i sprawdzają działanie modelu. Czasem po prostu kod jest
uruchamiany na maszynie, a jakość modelu sprawdzana jest ręcznie za
pomocą Jupytera.

\subsection[take-away]{Take away}

Zanim zaczniesz proponować rozwiązania nieprogramistkom/nieprogramistom,
musisz 3 razy dobrze upewnić się co i jak robią. Procesy muszą z jednej
strony zapewniać stabilną pracę, a z drugiej nie mogą być przeszkodą i
utrudnieniem.

Nietypowe rozwiązania wprowadzane przez nieprogramistów z reguły mają
swój dobry powód.


\stoptext
