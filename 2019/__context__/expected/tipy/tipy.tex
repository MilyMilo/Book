\usemodule[pycon-yyyy]
\starttext

\Title{Tipy o typowaniu}
\Author{Grzegorz Kocjan}
\MakeTitlePage

\subsection[wprowadzenie]{Wprowadzenie}

Świat enterprise to wielkie projekty, niezatapialne systemy bankowe,
super wydajne platformy przetwarzające dane, znacznie więcej niż prosty
sklep internetowy. Przez wiele lat świat ten był kojarzony tylko z
językami silnie typowanymi, stabilność ma w nich kluczowe znaczenie.
Teraz Python zyskał nową broń w walce o te super projekty - typy, nie
takie \quotation{silne} ale potężne.

\section[po-co-nam-typy-w-wolnym-świecie]{Po co nam typy w wolnym
świecie?}

Python jest kojarzony z dużą swobodą. Możemy zrobić w nim wszystko, jak
tylko nam się podoba. Jednak \quotation{Z wielką mocą wiąże się wielka
odpowiedzialność}. Jeśli nasz system ma być rozwijany dłużej niż przez
kilka dni, to warto skorzystać z typów i część tej odpowiedzialności
zrzucić na automatyzacje. Unikniemy wtedy, znanego nam dobrze,
dreszczyku emocji podczas aktualizowania produkcji. Poniższy błąd już
nie pojawi się więcej w naszym monitoringu.

\starttyping
TypeError: list indices must be integers or slices, not str
\stoptyping

Jednak tu nie chodzi tu tylko o samą stabilność. Kod, który korzysta z
typowania, staje się dużo bardziej czytelny. Oczywiście nie na początku.
Gdy po raz pierwszy spojrzałem na pełni typowany kod Pythona byłem
przerażony. Nie byłem w stanie przeczytać co ten kod robi, tyle nowych
składni, symboli, istna magia. Jednak, jak człowiek już się przyzwyczai,
nie będzie powrotu. Kod z dobrym typowaniem staje się dużo
przyjemniejszy do pracy. Kluczowe słowo \quotation{dobrym}. Przedstawię
wam kilka zasad, jak dobrze wykorzystać typ w pythonie.

\subsection[pierwsze-kroki]{Pierwsze kroki}

\section[typowanie-funkcji]{Typowanie funkcji}

Zanim jednak przejdziemy do bardziej zaawansowanych tematów, zobaczmy
jak wygląda typowanie. W pythonie 3.5 dostaliśmy możliwość dodawania
adnotacji typów dla funkcji.

\starttyping
def greeting(name: str) -> str:
    return "Hello " + name
\stoptyping

Za parametrem pojawił się dwukropek (\type{:}), za nim definiujemy
właśnie typ. W tym przypadku jest to \type{str} czyli ciąg znaków. Typ
wartości zwracanej możemy zdefiniować po strzałce \type{->}. Tu także
mamy \type{str}.

\section[typowanie-zmiennych]{Typowanie zmiennych}

Na samych definicjach funkcji świat się nie kończy. Dlatego też w
Pythonie 3.6 dostaliśmy możliwość definiowania typów zmiennych. Składnia
jest taka sama jak w przypadku argumentów funkcji. Python to przecież
piękny i spójny język!

\starttyping
order_name: str = 'PyConPL!'
\stoptyping

\section[typy-są-leniwcami]{Typy są leniwcami}

Bardzo ważna rzecz, którą trzeba wiedzieć o typach, to że są to leniwe
bestie. Same z siebie nie robią kompletnie nic. Dla uproszczenia możemy
założyć, że intepreter Pythona zwyczajnie je ignoruje w czasie
wykonywania kodu. Oznacza to, że możemy mieć całkowicie błędne typy i
nie wpłynie to negatywnie naszą aplikację. To z jednej strony bardzo
dobra wiadomość, dla osób, które chcą wdrożyć adnotacje typów w
istniejącym projekcie, z drugiej jednak, może wprowadzać pewne
zamieszanie ponieważ poniższy kod się wykona:

\starttyping
def some_function() -> int:
    return "Hello!"
\stoptyping

\section[mypy]{Mypy}

Jeżeli chcemy zweryfikować poprawność adnotacji musimy skorzystać z
zewnętrznej biblioteki. Nie martwcie się, jest to oficjalna biblioteka i
nie jest dostarczana jednocześnie z Pythonem, tylko ze względu na
dynamiczny rozwój. Cały czas jest usprawniania, poprawiane są
sporadyczne błędy i działa co raz szybciej. Instalujemy ją jak każdą
inną bibliotekę:

\starttyping
pip install mypy
\stoptyping

Kiedy już zainstalujemy mypy możemy uruchomić komendę sprawdzającą nasz
kod:

\starttyping
➜ mypy our_test_code.py
our_test_code.py:2: error: Incompatible return value type (got "str", expected "int")
\stoptyping

Dopiero uruchomienie \type{mypy} sprawdza poprawność naszych typów. W
odpowiedzi dostajemy listę wszystkich błędów, które należy poprawić w
naszym kodzie. Na całe szczęście otrzymujemy także podpowiedzi jakich
typów powinniśmy byli użyć.

Ponieważ typy są leniwe, a jedyną metodą na ich weryfikacje jest
uruchamianie mypy, dobrze jest włączyć to sprawdzanie w proces naszych
testów. Możemy skonfigurować zarówno tox, jak i pytest do sprawdzania
spójności typów za pomocą mypy.

\subsection[jak-żyć-w-zgodzie-z-typami]{Jak żyć w zgodzie z typami}

\section[zbyt-ogólne-typy]{Zbyt ogólne typy}

Na początku korzystania z typowania w pythonie zaczęliśmy dodawać
adnotacje do istniejącego kodu, bez wprowadzania większych modyfikacji.
To oczywiście dobre podejście, nie chcemy robić dwóch rewolucji w kodzie
w jednej chwili. Jednak popełniliśmy z tego powodu jeden znaczący błąd.
W wielu przypadkach nasze typy wyglądały tak:

\starttyping
def get(order_id: int) -> dict:
    ...
\stoptyping

Co taki słownik, jako wartość zwracana, mówi nam o pobieranym elemencie?
Poza tym, że to słownik, to niewiele. May dokładnie zero informacji o
tym jakich kluczy możemy się spodziewać w tym słowniku. Nie wiemy jakie
pola są wymagane, a jakie opcjonalne. Co gorsza, nie mamy także nad tym
kontroli. W tym słowniku możemy dodać dowolny, błędy z punktu
biznesowego, klucz i wszystko będzie \quotation{ok}.

Możemy poprawić czytelność tego kodu prostym zabiegiem. Wykorzystując
alias

\starttyping
from typing import Dict, Union
Order = Dict[str, Union[int, str]]
    
def get(order_id: int) -> Order:
    ...
\stoptyping

Teraz na pierwszy rzut oka wiemy czego możemy się spodziewać po naszym
słowniku. Powinien zawierać klucze i wartości związane z zamówieniem.
\quotation{Powinien}, ale wcale nie musi. Alias zwiększył czytelność
naszego kodu, ale nadal nie mamy kontroli nad tym co dokładnie znajduje
się w naszym słowniku. Zmierzamy powoli do pierwszego wniosku. Słownik
nie jest najlepszym typem z jakiego możemy skorzystać. Jeżeli chcemy
znacząco zwiększyć bezpieczeństwo naszej aplikacji powinniśmy skorzystać
z konstrukcji, które mają sztywno określone pola, np klasy. Jednak
tworzenie dużej ilości klas, tylko do przechowywania danych, w Pythonie
nie jest należy do najprzyjemniejszych. Na szczęście mamy kilka
ciekawych możliwości:

\startitemize[n,packed][stopper=.]
\item
  \type{dataclasses} - dostępne w standardowej bibliotece od pythona 3.7
\item
  \type{attrs} - zewnętrzna biblioteka, robi to co dataclasses, plus
  znacznie więcej
\item
  \type{pydantic} - kolejna biblioteka, jej zaletą jest wsparcie dla
  dataclasses
\stopitemize

Zobaczmy jak będzie wyglądać nasz prosty kod z wykorzystaniem
dataclasses:

\starttyping
@dataclass
class Order:
    id: int
    name: str

def get(order_id: int) -> Order:
    ...
\stoptyping

Definicja funkcji wygląda dokładnie tak samo, jednak tym razem zamiast
słownika będziemy zwracać obiekty klasy Order. To już będzie wymagało od
nas pewnych zmian w kodzie, jednak zyskujemy pewność co do pól
dostępnych w naszym zamówieniu. Nie możemy niczego przez przypadek
dodać, ani też zapomnieć o jednym z pól. Co jednak najprzyjemniejsze w
codziennej praktyce, nie musimy przeszukiwać setek linijek kodu, żeby
sprawdzić jak wygląda zamówienie.

\section[dom-dla-aliasów]{Dom dla aliasów}

Aliasy nie sprawdziły się nam do uzyskania odpowiedniej kontroli przy
zwracaniu złożonych danych. Zamiast aliasowanych słowników
wykorzystaliśmy, lepsze, bardziej dopasowane do sytuacji klasy. Czy to
oznacza, że aliasy są bezużyteczne? Nie! Aliasy są nam bardzo potrzebne
i bez nich typowanie często byłoby męczarnią. Najlepszym przykładem są
testy naszej aplikacji.

Czy testy też powinny być typowane? Tak! Początkowo, w projekcie w
którym zaczęliśmy dodawać typowanie, testy z nich nie korzystały. Jednak
po pierwszym błędzie w testach, który spowodował problemy na produkcji,
stwierdziliśmy jednogłośnie, że testy też powinny być typowane.

Wracając jednak do aliasów, wyobraźmy sobie fixture z pytest, które
zwraca nam statyczne dane w formie prostych implementacji interfejsów
domeny (więcej w temacie interfejsów domeny na mojej prezentacji).
Zwracana wartość takiego fixture będzie wyglądać następująco:

\starttyping
Tuple[IOrderRepository, IProductRepository, IClientRepository]
\stoptyping

Mając kilkanaście testów korzystających z tego fixture stracimy
jakąkolwiek widoczność tego co się dzieje.

\starttyping
def test_add_product_increases_order_total_cost(
    prepare_repositories=Tuple[
        IOrderRepository, IProductRepository, IClientRepository
    ]
) -> None:
    ....
def test_add_product_increases_order_items_count(
    prepare_repositories=Tuple[
        IOrderRepository, IProductRepository, IClientRepository
    ]
) -> None:
    ....
\stoptyping

Niemal połowę deklaracji naszego testu zajmują typy. Korzystając z nich
w taki sposób można się bardzo szybko zniechęcić. Jednak aliasy mogą
uratować czytelność naszego kodu.

\starttyping
StaticRepositories = Tuple[
    IOrderRepository, IProductRepository, IClientRepository
]

def test_add_product_increases_order_total_cost(
    prepare_repositories=StaticRepositories
) -> None:
    ....
def test_add_product_increases_order_items_count(
prepare_repositories=StaticRepositories
) -> None:
    ....
\stoptyping

Od razu lepiej! Aliasy świetnie się sprawdzają do poprawy czytelności
naszego kodu. Nie tylko zajmuje on miej miejsca, ale staje się bardziej
ekspresyjny, mówi sam za siebie.

\section[z-typami-pod-górkę]{Z typami pod górkę}

Wprowadzając typy do istniejącego, rozwijanego wiele lat projektu,
możecie natknąć się z sytuacją, że napisanie dobrych adnotacji będzie
bardzo trudne. To kolejna zaleta typowania! Przy każdym takim kawałku
kodu powinna się wam zapalić czerwona lampka. To będzie alarm i
jednocześnie znak, że czas na mały refactoring w tym miejscu. Wyobraźmy
sobie typ zwracanej wartości tej \quotation{pięknej} funkcji:

\starttyping
def book_meeting(customer_type):
    if customer_type == "regular":
        return {}
    elif customer_type == "vip":
        yield from (x for x in range(10))
    return None
\stoptyping

Przy sześciu linijkach od razu widać, że ten kod jest straszny. Jednak,
po 10 latach rozwijania projektu, zawsze znajdzie się taka magiczna
funkcja, która ma 300 linii i efektywnie robi coś równie ohydnego jak
powyższy przykład. Dzięki typowaniu, będziemy mogli łatwo namierzyć
takie podejrzane, niezbyt rozsądnie zaplanowane fragmenty.

\subsection[czy-powinienem-zacząć-korzystać-z-typów-w-pythonie]{Czy
powinienem zacząć korzystać z typów w Pythonie?}

Tak! Nie tylko poprawiają czytelność, ale także znacząco poprawiają
stabilność naszego kodu. Jeśli będziemy rozwijać nasz projekt dłużej niż
przez miesiąc, to warto.

\subsection[źródła]{Źródła}

\startitemize[n,packed][stopper=.]
\item
  Mypy, \useURL[url1][https://mypy.readthedocs.io/]\from[url1]
\item
  Pydantic,
  \useURL[url2][https://pydantic-docs.helpmanual.io/]\from[url2]
\item
  Attrs, \useURL[url3][https://www.attrs.org/]\from[url3]
\item
  migawka.it, \useURL[url4][http://migawka.it/]\from[url4]
\stopitemize


\stoptext
