\usemodule[pycon-yyyy]
\starttext

\Title{Wzorce projektowe w Pythonie}
\Author{Michał Mokrogulski}
\MakeTitlePage

\subsection[wprowadzenie]{Wprowadzenie}

Co, gdyby okazało się, że zupełnie inna osoba w zupełnie innym czasie i
przestrzeni logiki biznesowej stanęła przed dokładnie takim samym
problemem projektowym jak ty w tej chwili? Co, jeśli takich osób było
więcej niż jedna, a ich wiedza i doświadczenia są zebrane w jednym
miejscu i gotowe do użycia? Znajomość koncepcji rozwiązań powszechnych
problemów w połączeniu z zasadą 80/20 (20\letterpercent{} Twojej pracy
wystarczy na wygenerowanie 80\letterpercent{} rezultatów) może okazać
się bardzo wydajną bronią, która powinna być w rękach każdego
programisty.

\subsection[czym-są-wzorce-projektowe]{Czym są wzorce projektowe?}

W inżynierii oprogramowania wzorzec projektowy jest ogólnie powtarzalnym
rozwiązaniem często występującego problemu w projektowaniu
oprogramowania. Jest to opis lub szablon rozwiązania problemu, który
można wykorzystać w wielu różnych sytuacjach. Wzorce nie udostępniają
gotowego kodu, a jedynie ogólne sposoby rozwiązywania problemów w fazie
projektowania. Należy je samodzielnie zaimplementować w konkretnej
aplikacji.

\subsection[wspólny-słownik]{Wspólny słownik}

O ile łatwiej jest dogadać się z kimś, używając wspólnego słownika, nie
zaczynając wszystkiego od Adama i Ewy. Pozwoli to nie tylko na
przekazanie większej ilości informacji mniejszą ilością słów, ale
również pomoże myśleć o architekturze aplikacji bardziej abstrakcyjnie
na poziomie wzorca, a nie konkretnie na poziomie obiektu.

\subsection[wynalazek-czy-odkrycie]{Wynalazek czy odkrycie?}

Skąd wzięły się wzorce projektowe? Czy za każdym ze wzorców stoi
genialny autorytet, który wytyczył drogę rozwiązywania danego problemu?
Czy jest to praktyczna odpowiedź inżynierii, na zadany problem, która
została sprawdzona w boju? Krokiem milowym w dziedzinie wzorców
projektowych jest książka {\em Design Patterns: Elements of Reusable
Object-Oriented Software} autorstwa Bandy Czterech (GoF), w której
został wykonany ogrom pracy polegającej na przyjrzeniu się różnym udanym
systemom i wyciągnięciu z nim koncepcji rozwiązań tych samych problemów,
a następnie nazwaniu ich i pogrupowaniu. Pokazuje to, że wzorce
projektowe rodzą się w praktycznym środowisku, a następnie są odkrywane.

\subsection[wzorce-bandy-czterech]{Wzorce Bandy Czterech}

W artykule i na prezentacji zostaną przedstawione wzorce w wyżej
wspomnianej {\em Design Patterns: Elements of Reusable Object-Oriented
Software,} książki co nie oznacza, że są to jedynie istniejące wzorce
(tylko w książce jest ich 23, zostanie omówionych 6).

\section[strategia]{Strategia}

Definiuje rodzinę algorytmów, pakuje je jako oddzielne klasy i powoduje,
że są w pełni wymienne. Zastosowanie tego wzorca pozwala na to, aby
zmiany w implementacji algorytmów przetwarzania były całkowicie
niezależne od strony klienta.

\starttyping
import abc


class Context:
    def __init__(self, strategy):
        self._strategy = strategy

    def context_interface(self):
        self._strategy.algorithm_interface()


class Strategy(metaclass=abc.ABCMeta):
    @abc.abstractmethod
    def algorithm_interface(self):
        pass


class ConcreteStrategyA(Strategy):
    def algorithm_interface(self):
        pass


class ConcreteStrategyB(Strategy):
    def algorithm_interface(self):
        pass


def main():
    concrete_strategy_a = ConcreteStrategyA()
    context = Context(concrete_strategy_a)
    context.context_interface()
\stoptyping

Dzięki kompozycji możemy zmieniać zachowanie obiektu w czasie działania
programu tak długo, jak długo obiekty, których używamy do kompozycji,
będą implementować dany interfejs.

\section[stan]{Stan}

Umożliwia obiektowi zmianą zachowania wraz ze zmianą jego wewnętrznego
stanu. Po zmianie funkcjonuje on jako inna klasa. Wzorzec ten
hermetyzuje stan obiektu w odrębnych klasach, delegując do nich
odpowiedzialność obsługi konkretnych zdarzeń. Funkcjonowanie jako inna
klasa jest realizowane poprzez kompozycję i odwoływanie się do różnych
obiektów stanu.

\starttyping
import abc

class Context:
    def __init__(self, state):
        self._state = state

    def request(self):
        self._state.handle()


class State(metaclass=abc.ABCMeta):
    @abc.abstractmethod
    def handle(self):
        pass


class ConcreteStateA(State):
    def handle(self):
        pass


class ConcreteStateB(State):
    def handle(self):
        pass


def main():
    concrete_state_a = ConcreteStateA()
    context = Context(concrete_state_a)
    context.request()
\stoptyping

Warto zwrócić uwagę, że zastosowanie wzorca Stanu prowadzi do
zwiększenia liczby klas w projekcie, jest to cena za elastyczność i
jeśli przewidujemy zwiększającą się liczbę klas, będzie to rozwiązanie
zawsze korzystne. Pocieszeniem jest, że klienci nie wchodzą nigdy w
bezpośrednią reakcję ze stanami a jedynie przez Kontekst.

\section[singleton]{Singleton}

Wzorzec zapewniający, że klasa będzie miała tylko i wyłącznie jedną
instancję obiektu i zapewnia globalny punkt dostępu do tej instancji.

\starttyping
class Singleton(type):
    def __init__(cls, name, bases, attrs, **kwargs):
        super().__init__(name, bases, attrs)
        cls._instance = None

    def __call__(cls, *args, **kwargs):
        if cls._instance is None:
            cls._instance = super().__call__(*args, **kwargs)
        return cls._instance


class MyClass(metaclass=Singleton):
    pass


def main():
    m1 = MyClass()
    m2 = MyClass()
    assert m1 is m2
\stoptyping

Można byłoby zadać pytanie, czy nie wystarczyłoby użyć do tego zmiennych
globalnych? Zwróćmy uwagę, że wzorzec ma 2 założenia: 1 - zapewnienie
istnienia tylko jednej instancji obiektu 2- zapewnienie globalnego
punktu dostępu zmienne globalne mogą zapewnić realizację drugiego z tych
postulatów, ale nie pierwszego.

\section[metoda-fabrykująca]{Metoda Fabrykująca}

Definiuje interfejs pozwalający na tworzenie obiektów, ale pozwala klasą
podrzędnym decydować jakiej klasy obiekty zostanie stworzony. Wzorzec
ten przekazuje więc za tworzenie obiektów do klas podrzędnych.

\starttyping
import abc

class Creator(metaclass=abc.ABCMeta):
    def __init__(self):
        self.product = self._factory_method()

    @abc.abstractmethod
    def _factory_method(self):
        pass

    def some_operation(self):
        self.product.interface()


class ConcreteCreator1(Creator):
    def _factory_method(self):
        return ConcreteProduct1()


class ConcreteCreator2(Creator):
    def _factory_method(self):
        return ConcreteProduct2()


class Product(metaclass=abc.ABCMeta):
    @abc.abstractmethod
    def interface(self):
        pass


class ConcreteProduct1(Product):
    def interface(self):
        pass


class ConcreteProduct2(Product):
    def interface(self):
        pass


def main():
    concrete_creator = ConcreteCreator1()
    concrete_creator.product.interface()
    concrete_creator.some_operation()
\stoptyping

Wygląda na to, że jedyne co robimy, to przenosimy odpowiedzialność
tworzenia obiektów do innych podklas. I tak jest to prawda,
hermetyzujemy miejsce, które podlega zmianom, które może mieć wielu
klientów dzięki temu mamy tylko jeden element, który będzie podlegał
modyfikacjom podczas zmian w systemie.

\section[dekorator]{Dekorator}

Po pierwsze wzorzec Dekorator nie ma nic wspólnego z dekoratorami, czyli
natywną właściwością języka Python.

Pozwala na dynamiczne przydzielanie danemu obiektowi nowych zachowań,
dekoratory dają elastyczność do tej, jaką dają dziedziczenie, oferują w
zamian znacznie rozszerzoną funkcjonalność dodawania funkcjonalność w
sposób dynamiczny.

\starttyping
import abc

class Component(metaclass=abc.ABCMeta):
    @abc.abstractmethod
    def operation(self):
        pass


class Decorator(Component, metaclass=abc.ABCMeta):
    def __init__(self, component):
        self._component = component

    @abc.abstractmethod
    def operation(self):
        pass


class ConcreteDecoratorA(Decorator):
    def operation(self):
        self._component.operation()


class ConcreteDecoratorB(Decorator):
    def operation(self):
        self._component.operation()


class ConcreteComponent(Component):
    def operation(self):
        pass

def main():
    concrete_component = ConcreteComponent()
    concrete_decorator_a = ConcreteDecoratorA(concrete_component)
    concrete_decorator_b = ConcreteDecoratorB(concrete_decorator_a)
    concrete_decorator_b.operation()
\stoptyping

Obiekty dekorujące są tego samego typu co obiekty dekorowane, obiekt
podstawowy może zostać zawinięty w jeden lub w większą ilość
dekoratorów. Dekorator dodaje swoje zachowanie przed lub po delegowaniu
do obiektu dekorowanego właściwego zadania.

\section[fasada]{Fasada}

Zapewnia jeden, zunifikowany interfejs dla całego zestawu interfejsów
określonego podsystemu. Fasada tworzy nowy interfejs wysokiego poziomu,
który powoduje, że korzystanie z całego podsystemu staje się łatwiejsze.

\starttyping
class Facade:
    def __init__(self):
        self._subsystem_1 = Subsystem1()
        self._subsystem_2 = Subsystem2()

    def operation(self):
        self._subsystem_1.operation1()
        self._subsystem_1.operation2()
        self._subsystem_2.operation1()
        self._subsystem_2.operation2()


class Subsystem1:
    def operation1(self):
        pass

    def operation2(self):
        pass


class Subsystem2:
    def operation1(self):
        pass

    def operation2(self):
        pass


def main():
    facade = Facade()
    facade.operation()
\stoptyping

Podczas projektowania i tworzenia systemu powinniśmy zwracać szczególną
uwagę na liczbę klas współpracujących z sobą i ograniczać interakcję
tylko do potrzebnego minimum. Pozwala to uniknąć sytuacji, w których
wiele różnych klas jest z sobą ściśle powiązanych, a zmiana w jednej
klasie powoduje wiele zmian w innych klasach.

\subsection[źródła]{Źródła}

\startitemize[n,packed][stopper=.]
\item
  \useURL[url1][https://sourcemaking.com/]\from[url1]
\item
  \quotation{Rusz głową Wzorce projektowe}, autorzy: Eric Freeman, Bert
  Bates, Kathy Sierra, Elisabeth Robson
\item
  \quotation{Design Patterns: Elements of Reusable Object-Oriented
  Software}, autorzy: Erich Gamma, Richard Helm, Ralph Johnson, John
  Vlissides
\stopitemize


\stoptext
