\usemodule[pycon-yyyy]
\starttext

\Title{Agile mindset, agile code}
\Author{Piotr Podgórski}
\MakeTitlePage

\subsection[abstract]{Abstract}

Introducing agility into an organization is not an easy task, not matter
if you do it in a stale organization or a completely fresh one.

What's to blame? Our inherent nature, for one, as we are creatures of
hierarchy. Self-organization is not as natural to us as we would like to
think. But equally important are the misconceptions about agile in
general and the lack of appreciation for software patterns.

Let's look at those misconceptions, the culture of agility, and what
{\em processes and tools} to apply to the code once we've transformed
the {\em individuals and interactions}.

\subsection[lets-do-agile]{Let's do Agile!}

What does introducing agile to a company usually look like? A manager
learns of a methodology which speeds up development and requires less
up-front planning, called \quotation{Agile}. They assume it saves money,
produces more features and Makes Development Great Again. So obviously
they want to jump on board.

Let's say they go online and try to look for resources on the subject.
Chances are they come out with a very strong conviction that
\quotation{doing Agile} effectively means Scrum, and Scrum, in turn,
means sprints. Before you know it, a JIRA license is bought, someone is
titled Product Owner and they begin stuffing all the current tickets in
a backlog, creating a {\em five year plan} cut up in 2-week packets.

Given that Scrum is a process (actually a framework, but for the sake of
argument) and JIRA is a tool, stack that against {\em The Agile
Manifesto's} first and foremost point: \quotation{Individuals and
interactions over processes and tools} and you know something's
backwards. An agile revolution is a change of who you are, not of what
you do. \quotation{Agile} is an {\em adjective}. \quotation{Doing agile}
makes about as much sense semantically as \quotation{doing purple}.

\subsection[a-lesson-from-general-motors]{A lesson from General Motors}

Between the late 1940s and 1970s Toyota came up with a process now
called the TPS - Toyota Production System. It allowed them to produce
better quality cars at a much lower cost than anyone thought possible,
so obviously American manufacturers tried to imitate it.

In fact, Toyota encouraged it and gave tours of it's factories. They
knew you can copy the tools and the processes, but it's the culture that
makes them work. And you can't copy that culture just by looking at
people screwing cars together. Plus, Toyota's agile mentality also meant
they would've improved on the processes before their competitors landed
back in the US.

Part of the TPS is a tool - the Andon Cord. Whenever an assembly line
worked notices a quality issue, they pull on that cord and the line
stops. At that point engineers gather around and figure out how to solve
the issue, before you have a lot of defective cars no-one wants to buy.
Or, in other words, before you become General Motors of that time.

The cord worked in Japan and it was one of the things GM copied. After a
while, they were surprised to find that the TPS is crap. It cost more,
made less cars and the quality was just as bad as ever. Yeah, well, they
kind of forgot the cord doesn't work if people are afraid to pull it.

You see, GM had an authoritative, hierarchical structure. The engineers
were there to design the cars and the peasants at the assembly line were
there to put them together. If something didn't fit, it was a peasant
problem and the engineers were not to be bothered with such mundane
matters.

The whole company was focused on utilization - making the most of their
work force and getting as many cars of the line as possible. This
doesn't resonate with the idea of stopping the assembly line until you
figure out a way to not have Tesla Model 3 panel gaps on your cars. You
can't be agile if you're afraid to slow down.

\subsection[the-legacy-of-low-agility]{The legacy of low agility}

Consider a software system built by a \quotation{1970s GM of software}.
High pressure, impossible deadlines, badly understood business
requirements and utilization as a measurement of productivity. That in
mind, let me introduce you to the {\em Conway's law:}

\startblockquote
organizations which design systems \ldots{} are constrained to produce
designs which are copies of the communication structures of these
organizations.
\stopblockquote

That software will have an accidental architecture, molded by all the
shortsighted time- and cost-saving decisions. It won't have an automated
deployment pipeline or test coverage, because ain't nobody got time for
that. It's gonna be a mess.

Let's now put effort (and sweat and tears) into reshaping the
organization so it has a flat structure full of trust. Imagine its
culture turned into a 1970s Toyota with a management process as lean and
agile as a Parkour runner. Bad news, though - it still won't deliver a
single sprint.

Why? Because you now have a very agile organization working with legacy
code, written in a non-agile era. Everyone's eager to pull on the
{\em adon cord}, but there's none and they're actually stuck at a Ford
Model T-style production line.

Luckily, having the right culture (a better guide to getting there is
beyond the scope of this text, sorry), you can start using the right
tools and processes to bring your code into the new era. But first,
let's try to understand how that big ball of mud we call
\quotation{legacy code} came to be in the first place.

\subsection[short-term-benefits-long-term-headaches]{Short term
benefits, long term headaches}

Let me use an example of a decision made early on, which can have dire
consequences. Using the ORM as your domain entities and an SQL database
as your domain model.

At first, this sounds reasonable enough that people tend to accept it
with little thought. It's actually what tutorials tell you to do. And
while it's good enough for simple projects (like a blog), these are some
very serious assumptions to make up front, which are difficult to back
away from.

After a while you may consider getting your data from places other than
SQL. NoSQL document storage, a graph database, an object database or
even some entirely external service. Moreover, a single domain entity
could be sourced from a combination of these, but doing that without
layers of indirection (when ORM is your business logic) requires working
around your own architecture. Get a large enough project and the problem
will only grow with time.

This is where accidental complexity comes from - starting with a locked
down decision and trying to work around it. That is the exact opposite
of agility.

\subsection[digging-yourself-out]{Digging yourself out}

Ok, but how do we make an agile revolution happen in legacy code built
by a non-agile company?

{\em Gradually.}

Of course you could just create a new repo and start rewriting the whole
thing, but that's called the {\em second system syndrome}. Possible
outcomes?

\startitemize[n,packed][stopper=.]
\item
  it never reaches production and you just stay with the old system
\item
  your organization goes bankrupt because it could not keep up with the
  competition
\item
  you start to rush it at some point and end up with even more mess.
\stopitemize

So that won't work. We're stuck with re-shaping the existing system. For
that, let's turn to David Wheeler's {\em fundamental theorem of software
engineering}:

\startblockquote
All problems in computer science can be solved by another level of
indirection
\stopblockquote

And follow the advice the advice from Kent Beck, so we don't try to do
everything at once:

\startblockquote
for each desired change, make the change easy (warning: this may be
hard), then make the easy change
\stopblockquote

The procedure to follow is not so different from the standard TDD
{\em Red-Green-Refactor} cycle.

\subsection[test-coverage]{Test coverage}

Face the facts - it doesn't have tests. In fact, looking at legacy
projects I find an interesting correlation - the less important a piece
of code, the more test coverage. Some irrelevant CSV exporter?
100\letterpercent{} branch coverage. The master function doing
everything from user log in to firing nuclear missiles? A single test.
For a happy path. And it's probably commented because it didn't work.

It will be ugly, but there's no other way - make sure you cover as much
business logic as possible. Take that opportunity to learn the system.
It might even be impossible to tell what states it may have, so operate
under the assumption it runs on magic. Preparing the {\em interactive
specification} (aka the test suite) will let you understand all the
quirks and notice the areas for improvement.

However, do not attempt to fix or refactor anything at this point.
Seriously, don't. Only once you're satisfied with the test suite can you
try to renovate the whole thing.

\subsection[code-for-deletion]{Code for deletion}

Feature flags we usually associate with A/B testing and relatively small
changes like the placement or shape of a button. In reality, it's a
great tool to increase the agility of your code.

Using feature flags you can make serious changes gradually, while
pushing them continuously to production. It limits the risks involved
with deploying changes and helps you avoid merge conflicts and work
better as a team - all great things if you want to score agility points.

Most importantly though, it also shapes the way you (re)build the code.
It will force you to increase cohesion and reduce coupling. Eventually,
you will just replace code instead of modifying it, which is the
ultimate goal when it comes to code agility.

If you now think \quotation{what about changing the data base schema, I
can't run two schemas at the same time!} You're right, you (usually)
can't. But your business logic shouldn't really care about your database
schema, only about entities. That's what the {\em \quotation{Short term
benefits, long term headaches}} section was about. Choosing to support
feature flags will, again, motivate you to reduce that coupling.

A cool side-effect of being able to run two versions side by side is you
can make sure they do the same thing before making a switch. Once you're
confident, you just scrap the old one - leaving only the new and shiny
code. Stress reduction is always a nice bonus.

\subsection[plumbing]{Plumbing}

Another great way of learning new things about your system and forcing
yourself to think of how it should be structured is plumbing - working
on your pipelines.

Tiny, throw-away systems obviously don't need the whole
\quotation{continuous deployment with automated horizontal scaling
architecture as code} shebang, but if you got this far in this text,
yours probably does (or you're just curious, which is also great).

Among other things, it helps you understand the value of configuration.
Most programmers don't really consider it {\em code,} yet a config file
is just as much a piece of code as anything else. You realize that as
soon as you have to push everything through a pipeline in a tractable
and secure way.

\subsection[set-the-stage]{Set the stage}

Staging. If you don't have one - get one. If you do, ask yourself a very
serious question: is it really? Many staging environments I've seen are
in fact glorified testing servers. Either they don't resemble the
production infrastructure at all or they're full of old code which never
ended up on production. Or both.

Since you're already looking at your staging, take a look at your
repository as well. Got some unfinished or un-deployed code on staging?
Then your workflow probably looks something like this:

\starttyping
    * Merge feature into production
    |\
    |  \
    | * | Merge feature into staging
    | |\|
    | | * feature
    | | |
    * |/ ...
    |/|
\stoptyping

Staging is meant to be a stop on your path to production, just as it is
in QA. Yet in the repo stuff is merged into production without going
{\em through} staging, which means staging will drift away - there's
really nothing blocking you from creating a huge branch with a multitude
of changes, merging it into staging and leaving it there indefinitely.
Or merging into production without going through staging.

A better alternative? There are numerous options, but you generally
don't want to merge your feature branches into more than one place. For
instance:

\starttyping
    * Merge feature into production
    |\
    | * Merge feature into staging
    | |\
    | | * feature
    | * | ...
    | |/
    | |
\stoptyping

Again, the purpose of this is to make it {\em inconvenient to do the
wrong thing}.

Made a huge, conflict-inducing branch and merged it into staging? You
get punished by a freeze on your entire pipeline. Again, this workflow
promotes incremental changes and small PRs. It amplifies the need for
feature flags. It forces you to change your thinking from short term
convenience to {\em sustainability}.

Ideally, you get to a point where your feature branches live for a day
or two before they're merged, and they build on each other, making code
review a breeze. Another huge benefit.

\subsection[suggestions-not-commandments]{Suggestions, not commandments}

While I believe following these rules is beneficial, context matters! As
Kevlin Henney likes to put it, looking left when crossing a one-way
street is a great rule in continental Europe but a quick way to get
yourself killed in the UK. Same with this.

The point of Clean Architecture (and the likes) is to push important
decisions as far away as possible. Sometimes, though, the most important
decision is whether or not you need these rules at all, because they do
come at a cost. Sure, they help you avoid accidental complexity, but
intentional complexity is just as bad when it serves no purpose.

Software design patterns are there for a reason, but just like Scrum,
Kanban and the TPS, they don't exist in vacuum. They're there to help
you make sense of a large system, but they can just as easily
{\em create} one. If you don't know what I mean, just google
\quotation{fizzbuzz enterprise edition}.

\subsection[pain-is-good-pain-means-youre-alive]{Pain is good, pain
means you're alive}

You're probably noticing that most of these techniques are there to
force you to do something painful. This is intentional. If you look at
the history of programming, it's basically defined by taking away
freedoms. Just like that time we got rid of \type{goto}.

All of these things make writing individual lines of code {\em harder},
so writing systems and collaborating can be {\em easier}. In the words
of Mark Seemann:

\startblockquote
Some developers seem to think that typing is a major bottleneck while
programming. It's not.
\stopblockquote

The bottleneck is understanding what to type and what's been typed.
That's exactly why you should be wary of everything that saves you a few
minutes, because more often than not, it's also creating a time bomb. On
the other hand, there's a good change that processes and tools which
require a lot of work/learning at first, will benefit you in the long
run.

\subsection[sources]{Sources}

\startitemize[n,packed][stopper=.]
\item
  Martin, Robert C. (2017). Clean Architecture: A Craftsman's Guide to
  Software Structure and Design
\item
  Jez Humble, Barry O'Reilly, Joanne Molesky (2014). Lean Enterprise:
  How High Performance Organizations Innovate at Scale
\item
  Kevlin Henney,
  \useURL[url1][https://www.youtube.com/playlist?list=PL6wxfKvkNqRugfIiKKgRXa_0wKIQW_ZEH]\from[url1]
\item
  Greg Young - The art of destroying software,
  https://vimeo.com/108441214
\stopitemize


\stoptext
