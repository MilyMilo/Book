\usemodule[pycon-yyyy]
\starttext

\Title{Python 3.8 - hit, czy shit?}
\Author{Marcin Bardź}
\MakeTitlePage

\subsection[streszczenie]{Streszczenie}

Artykuł opisuje nowości, jakie przynosi Python w wersji 3.8, kładąc
szczególny nacisk na konsekwencje wprowadzenia operatora przypisania,
który to stał się kością niezgody wewnątrz pythonowej społeczności.
Postaram się też spojrzeć na rozwój Pythona w szerszej perspektywie i
zastanowić się nad jego bliższą oraz dalszą przyszłością.

\subsection[wprowadzenie]{Wprowadzenie}

Python nieustannie zmienia się i ewoluuje - przez ponad ćwierć wieku
istnienia, przeobraził się nie do poznania, dojrzał i ogromnie
się~rozwinął. Zaczynał jako niszowa alternatywa do skryptów powłoki,
dziś jest popularnym i wszechstronnym językiem ogólnego przeznaczenia o
szerokim wachlarzu zastosowań.

Drogę tę~przebył głównie dzięki mądrym decyzjom społeczności pod
przewodnictwem Guido von Rossuma, społeczności która czuwała, by rozwój
Pythona odbywał się zawsze we właściwym kierunku.

I wtedy, gdy wszystko szło tak dobrze, pojawił się PEP 572 - czyli
złowrogi operator przypisania, który podzielił społeczność do tego
stopnia, że sam Ojciec Założyciel postanowił zejść z tronu i zrzekł się
swojej funkcji BDFL ({\em Benevolent Dictotor For Life}), pozostawiając
swoje dziecko oraz społeczność samym sobie.

Od tego czasu minął już rok, a na dniach wydany zostanie Python 3.8, w
którym premierę będzie miał operator przypisania. Jaka przyszłość czeka
osieroconego Pythona? Czy jest to punkt zwrotny w życiu naszego
ulubionego języka? Czy społeczność poradzi sobie z brzemieniem
odpowiedzialności? Na żadne z powyższych pytań nie odpowiem w niniejszym
artykule, ale postaram się przedstawić kilka faktów i spekulacji
związanych z najnowszym Pythonem.

\subsection[kość-niezgody-czyli-pep-572]{Kość niezgody, czyli PEP 572}

Operator przypisania \type{:=} ({\em walrus operator}, ang. {\em walrus}
= mors) ma na celu uproszczenie kodu przez umożliwienie przypisywania
wartości do zmiennych wewnątrz wyrażeń. Istnieje on w wielu językach,
choćby w tych C-podobnych, gdzie niewątpliwie wyrządził wiele szkód.
Wersja pythonowa ma być bezpieczniejsza, gdyż używa dedykowanego symbolu
(\type{:=}), odróżniającego operator od instrukcji przypisania
(\type{=}), ponadto dość~mocno ograniczone są miejsca, gdzie operator
może być on użyty.

Oto kilka przykładów zastosowania operatora przypisania zaczerpniętych z
PEP 572:

\starttyping
# Obsługa wyrażeń regularnych
if (match := pattern.search(data)) is not None:
    x = match.group(1)

# Pętla trudna do napisania w inny sposób
while chunk := file.read(8192):
   process(chunk)

# Ponowne wykorzystanie kosztownych obliczeń
[y := f(x), y**2, y**3]

# Współdzielenie wyrażenia w filtrze i w rezultacie dopełnienia
filtered_data = [y for x in data if (y := f(x)) is not None]
\stoptyping

Wygląda to ładnie i zgrabnie, ale największymi kontrowersjami związanymi
z \quotation{morsem} są możliwość~niewłaściwego użycia i potencjalne
zaciemnienie kodu. Przyjrzyjmy się więc mniej ładnym przykładom:

\starttyping
(x := 1)  # Nawias wymagany, równoważne z: x = 1
(x = (y := (z := 1)))  # Ekwiwalent: x = y = z = 1
(x := 1, 2)  # Wartość x wynosi 1, ':=' ma wyższy priorytet, niż ','
(x := (1, 2))  # Tym razem x = (1, 2)
f(p=(x := 1))  # To proste: x = 1; f(p=x)
(a.b := 1)  # Nie działa
(a[1] := 1)  # Też nie działa
\stoptyping

Jak widać, kod nie zawsze zyskuje na czytelności, a funkcjonowanie
nowego operatora potrafi czasem zaskakiwać i niewątpliwie wprowadzi do
języka nową kategorię trudnych do wyłapania błędów. Dodatkową kwestią
jest fakt, że zmienne są definiowane w nadrzędnej przestrzeni nazw,
prowadząc do jej zaśmiecenia, co może prowadzić na przykład do
niezamierzonego nadpisania wartości zmiennej. Ponadto, pojawiła się
alternatywna składnia, częściowo duplikująca już istniejącą
funkcjonalność, co jest jawnie~sprzeczne z przykazaniami Zen Pythona.

Z drugiej strony czasem nasz \quotation{mors} potrafi drastycznie
zwiększyć czytelność kodu przez likwidację sztucznych zagnieżdżeń. Oto
przykład z biblioteki standardowej, z pliku \type{copy.py}. Wersja
\quotation{tradycyjna}:

\starttyping
reductor = dispatch_table.get(cls)
if reductor:
    rv = reductor(x)
else:
    reductor = getattr(x, "__reduce_ex__", None)
    if reductor:
        rv = reductor(4)
    else:
        reductor = getattr(x, "__reduce__", None)
        if reductor:
            rv = reductor()
        else:
            raise Error(
                "un(deep)copyable object of type %s" % cls)
\stoptyping

Oraz wersja usprawniona dzięki operatorowi przypisania:

\starttyping
if reductor := dispatch_table.get(cls):
    rv = reductor(x)
elif reductor := getattr(x, "__reduce_ex__", None):
    rv = reductor(4)
elif reductor := getattr(x, "__reduce__", None):
    rv = reductor()
else:
    raise Error("un(deep)copyable object of type %s" % cls)
\stoptyping

Czy Python potrzebował PEP 572? Trudno powiedzieć, ale na pewno
konsekwencje wojny wokół tego dokumentu są bardzo poważne i wpłyną
długofalowo na przyszłość języka.

\subsection[co-ponadto-w-wersji-3.8]{Co ponadto w wersji 3.8?}

Nowy Python, to nie tylko PEP 572 - mamy szereg zmian, zarówno w samym
języku, jak i w bibliotece standardowej. Poniżej pokrótce omówię
niektóre z nich.

\section[argumenty-wyłącznie-pozycyjne]{Argumenty wyłącznie
pozycyjne}

Moduły Pythona napisane w C już od dawna mają możliwość~wymuszenia
użycia argumentów funkcji wyłącznie przez ich nazwę, jak i wyłącznie
pozycyjnych. Dla modułów napisanych w Pythonie te możliwości były
ograniczone i nieraz zmuszały twórców bibliotek do manualnej weryfikacji
już w ciele funkcji.

W Pythonie 3.0 pojawiła się możliwość~deklaracji parametrów wyłącznie
nazwanych przez wstawienie \type{*} do listy argumentów - wszystkie
parametry po prawej stronie gwiazdki wymagały podania ich nazwy przy
wywołaniu. Oto przykład:

\starttyping
def f(a, b, *, c):
    pass

f(1, 2, c=3)  # OK
f(a=1, b=2, c=3)  # OK
f(1, 2, 3)  # BŁĄD!
\stoptyping

Jednakże wszystkie agrumenty przed \type{*} mogły być użyte zarówno
pozycyjnie, jak i przez nazwę, co nie zawsze było zachowaniem pożądanym.
Dlatego PEP 570 wprowadza nowy symbol \type{/} w liście argumentów,
oddzielający parametry wyłącznie pozycyjne (po jego lewej) od tych
pozycyjno/nazwanych (po prawej). Tak więc w Pythonie 3.8 możemy już
napisać:

\starttyping
def f(a, /, b, *, c):
    pass

f(1, 2, c=3)  # OK
f(a=1, b=2, c=3)  # BŁĄD!
f(1, 2, 3)  # BŁĄD!
\stoptyping

Co więcej, zapis używający \type{*} i \type{/} pojawia się~już od
jakiegoś czasu w dokumentacji samego Pythona, jak i w niektórych
bibliotekach (\type{numpy}). Spójrzmy co nam powie \type{help()} dla
paru funkcji wbudowanych. Rozszyfrowanie zapisu nie powinno stanowić
trudności dla uważnego czytelnika:

\starttyping
>>> help(len)
len(obj, /)
...
>>> help(sorted)
sorted(iterable, /, *, key=None, reverse=False):
...
\stoptyping

\section[równoległy-cache-dla-skompilowanych-plików]{Równoległy
cache dla skompilowanych plików}

Kolejnym miłym usprawnieniem w Pythonie 3.8 jest możliwość wyrzucenia
skompilowanych plików (\type{__pycache__}) z katalogów kodu źródłowego.
Służy do tego nowa zmienna środowiskowa \type{PYTHONPYCACHEPREFIX} i
parametr wiersza poleceń \type{-X pycache_prefix}.

\section[znak-w-łańcuchach-formatujących]{Znak \type{=} w
łańcuchach formatujących}

Nowy modyfikator f-łańcuchów produkuje zapis w postaci
\type{nazwa=wartość} i może oszczędzić nieco czasu programistom:

\starttyping
>>> a = 1
>>> f"{a=}"  # Samoopisująca się wartość
'a=1'
>>> f"{(mors:=1)=}"  # O to też zadbano
'(mors:=1)=1'
\stoptyping

\section[pickle---protokół-w-wersji-5]{Pickle - protokół w wersji
5}

W nowej wersji protokołu wprowadzono wsparcie dla zewnętrznych buforów
(ang. {\em out-of-band}), pomocnych przy przetwarzaniu wielordzeniowym i
wielomaszynowym. Umożliwia ono optymalizację przesyłu danych przez
eliminację niepotrzebnych kopii danych w pamięci, a także daje
możliwość~zastosowania specjalizowanych algorytmów kompresji.

\section[pozostałe-zmiany]{Pozostałe zmiany}

Poza tym, co widoczne, mamy do czynienia ze szeregiem zmian
niezauważalnych dla przeciętnego użytkownika (PEP 578, PEP 587, PEP
590), drobnych poprawek składni, małych, ewolucyjnych zmian w
bibliotekach oraz mnóstwem optymalizacji.

\subsection[python-3.8---hit-czy-shit-1]{Python 3.8 - hit, czy shit?}

Czy operator przypisania zmieni wiele w Pythonie? Prawdopodobnie nie,
największa zmiana z nim związana już się dokonała, kiedy to zaszczuty
Guido postanowił usunąć się w cień. Czy będę używać nowego operatora?
Oczywiście, że tak! My programiści jesteśmy z natury leniwi, więc każda
zaoszczędzona linijka to czysty zysk.

Poza nieszczęsnym \quotation{morsem}, wersja 3.8 nie wprowadza żadnych
rewolucyjnych zmian, wpisując się w przemyślany i stabilny schemat
rozwoju Pythona. Wszystkie wdrożone modyfikacje idą ku większej
prostocie, spójności i szybkości języka. Dlatego też jestem spokojny o
przyszłość naszego ulubionego cyfrowego gada.

\subsection[źródła]{Źródła}

\startitemize[n,packed][stopper=.]
\item
  Co nowego w Pythonie 3.8,
  \useURL[url1][https://docs.python.org/3.8/whatsnew/3.8.html]\from[url1]
\item
  PEP 572 - Assignment Expressions,
  \useURL[url2][https://www.python.org/dev/peps/pep-0572/]\from[url2]
\item
  PEP 570 - Python Positional-Only Parameters,
  \useURL[url3][https://www.python.org/dev/peps/pep-0570/]\from[url3]
\item
  PEP 574 - Pickle protocol 5 with out-of-band data,
  \useURL[url4][https://www.python.org/dev/peps/pep-0574/]\from[url4]
\item
  {[}python-committers{]} Transfer of power,
  \useURL[url5][https://www.mail-archive.com/python-committers@python.org/msg05628.html]\from[url5]
\stopitemize


\stoptext
