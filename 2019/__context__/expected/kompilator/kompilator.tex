\usemodule[pycon-yyyy]
\starttext

\Title{Kompilator w Pythonie}
\Author{Krzysztof Jura}
\MakeTitlePage

\subsection[wstęp]{Wstęp}

Zastanawiałeś się kiedyś jak działają kompilatory? A może myślałeś o
tym, aby napisać swój własny kompilator, ale nie wiedziałeś od czego
zacząć? W tym artykule przybliżę Ci temat kompilatorów, ich budowy oraz
pokażę w jaki sposób można napisać je w Pythonie wykorzystując narzędzie
ANTLR4.

\subsection[kompilator]{Kompilator}

Na początku należy wyjaśnić czym w ogóle jest kompilator. Powołując się
na polską Wikipedię, {\bf kompilator} to *program służący do
automatycznego tłumaczenia kodu napisanego w jednym języku (języku
źródłowym) na równoważny kod w innym języku (języku wynikowym)
\footnote{Wikipedia, \quotation{Kompilator},
  \useURL[url2][https://pl.wikipedia.org/wiki/Kompilator]\from[url2].}.

Językiem wynikowym może być np. kod asemblera lub kod maszynowy (dla
języków \footnote{W tym jak i innych nawiasach zaczynających się od
  \quotation{{\bf {\em dla języków}}} nie chodzi o języki, lecz o ich
  najpopularniejsze implementacje.}: C, C++), kod bajtowy (dla języków:
Python, Java, C\#), kod pośredni (np. C, LLVM IR) kompilowany do kodu
maszynowego (dla języków: wczesny C++, Rust), a nawet kod wysokiego
poziomu, będący na podobnym poziomie abstrakcji co kod źródłowy (np.
kompilacja TypeScript -> JavaScript). W tym ostatnim przypadku zamiast
używać terminów kompilator/kompilacja często używa się słów
transpilator/transpilacja.

Kompilator wykonuje proces kompilacji składający się z kilku kroków. W
zależności od kompilatora liczba kroków jak i same kroki mogą się
diametralnie różnić. W teorii kompilatorów wyróżnia się kilka
podstawowych kroków kompilacji \footnote{Liczba kroków kompilacji, jak i
  sam ich podział może różnić się w zależności od źródła. W niektórych
  kompilatorach przed fazą analizy leksykalnej można spotkać krok
  \quotation{{\em przetwarzania wstępnego}} (ang. preprocessing), a fazy
  optymalizacji mogą występować nawet kilkukrotnie, między różnymi
  krokami. Krok generowania formy pośredniej jest w zasadzie możliwy do
  ominięcia - można razu wygenerować kod wynikowy. W przypadku
  kompilatorów generujących kod bajtowy czasami uznaje się, że kod
  wynikowy jest tożsamy z formą pośrednią, a więc kroki 4 i 5 są jednym
  i tym samym krokiem.}:

\startitemize[n,packed][stopper=.]
\item
  Analiza leksykalna (tokenizacja)
\item
  Analiza składniowa/syntaktyczna (parsowanie)
\item
  Analiza semantyczna
\item
  Wygenerowanie formy pośredniej
\item
  Wygenerowanie kodu wynikowego
\stopitemize

W dalszej części artykułu zrobimy kompilator prostego, wymyślonego
języka. Nasz język nazwiemy {\bf Conpy} (anagram słowa PyCon). Język
będzie naprawdę minimalistyczny i wspierający tylko konstrukcje językowe
występujące w tym przykładzie:

{\bf test.conpy}

\starttyping
func test(a) {
    if (a) {
        print(a);
    } else {
        print(2 + 2 * 2);
    }
}
test(5);
\stoptyping

Jak widać, funkcje definiowane po słowie {\em func}, będą mogły mieć
pojedynczy parametr (lub brak parametrów). Bloki kodu będą wyznaczane za
pomocą nawiasów klamrowych. Oprócz tego w naszym języku istnieć będą
instrukcje warunkowe \type{if} wraz z opcjonalnymi blokami \type{else}.
Istnieć będzie też funkcja \type{print} wypisująca na ekran wyrażenie
podane w argumencie. Wyrażenia będą składać się z pojedynczej wartości
(literału całkowitoliczbowego lub parametru) lub z operacji
matematycznych ( dodawanie / mnożenie). Dodatkowo, po każdym wywołaniu
funkcji będziemy musieli stawiać średnik.

\subsection[zaczynamy]{Zaczynamy!}

Teoretycznie moglibyśmy kompilator naszego języka napisać w czystym
Pythonie bez korzystania z żadnych bibliotek zewnętrznych, ale nie
chodzi nam o to by na nowo wynaleźć koło, tylko o to, aby zrobić
kompilator szybko, sprawnie i przyjemnie. Do tego celu wykorzystamy
narzędzie {\bf ANTLR4}, dzięki któremu będziemy mogli wygenerować kod
Pythonowy obsługujący 2 pierwsze kroki kompilacji (analizę leksykalną i
składniową) i udostępniający mechanizmy dzięki którym kolejne kroki
kompilacji będą dużo łatwiejsze do zrealizowania. Minusem (bądź plusem w
zależności od punktu widzenia) jest to, że będziemy musieli mieć
zainstalowaną Javę, gdyż ANTLR4 jest właśnie w niej napisany. ANTLR4
instalujemy według instrukcji znajdujących się na początku dokumentacji,
do której link znajdziemy na oficjalnej stronie projektu:
\useURL[url3][https://www.antlr.org][][antlr.org]\from[url3]. Po
instalacji, gdy już będziemy mieli dostępną komendę {\bf antlr4}, musimy
jeszcze zainstalować Pythonowy pakiet {\bf antlr4-python3-runtime} .
Jeżeli już wszystko zainstalowałeś to możemy przejść do omówienia i
implementacji kroków kompilacji.

\subsection[analiza-leksykalna]{Analiza leksykalna}

Jest to etap wykonywany przez część kompilatora zwaną {\em lekserem} lub
{\em analizatorem leksykalnym}. Lekser dostaje na wejściu kod źródłowy i
generuje z niego strumień {\em tokenów}. Tokeny są obiektami
wygenerowanymi z fragmentów kodu źródłowego, posiadają one unikalne ID
oraz informacje o tym z jakiego fragmentu kodu zostały wygenerowane i w
którym miejscu w kodzie źródłowym ów fragment wystąpił. Przejdźmy teraz
do przykładu działania leksera; chcielibyśmy, aby nasz kompilator z
kodu:

\starttyping
func test(a) {
    print(a + 2 * 2);
}
\stoptyping

wygenerował następujący strumień tokenów (tokeny dla czytelności zostały
wypisane po przecinkach oraz w liniach odpowiadającym liniom kodu
źródłowego, z których zostały one wygenerowane):

\starttyping
FUNC, NAME, OPEN_PAREN, NAME, CLOSE_PAREN, OPEN_BRACKET
NAME, OPEN_PAREN, NAME, PLUS, INT, STAR, INT, CLOSE_PAREN, SEMICOLON
CLOSE_BRACKET
\stoptyping

Wynikowy strumień tokenów można następnie przekazać do analizy
składniowej, w której już nie będzie potrzeby zaglądać do kodu
źródłowego ponieważ wszystkie potrzebne informacje zostały już z niego
wyciągnięte i zapisane w przystępniejszej formie, czyli w strumieniu
tokenów. Zanim przejdziemy dalej, warto jeszcze dodać, że z białych
znaków oraz znaków nowej linii nie chcemy generować tokenów, gdyż znaki
te nie mają żadnego znaczenia w naszym wymyślonym języku, a powyższy kod
źródłowy równie dobrze można by rozbić na 10 linii lub zapisać w 1
linii, a wynik kompilacji (oraz wygenerowany strumień tokenów) nie
powinien ulec zmianie.

\subsection[analiza-składniowa-syntaktyczna]{Analiza składniowa /
syntaktyczna}

Jest to etap wykonywany przez część kompilatora zwaną {\em parserem} lub
{\em analizatorem składniowym}. Parser, według~ściśle określonych reguł,
z otrzymanego strumienia tokenów generuje drzewo wyprowadzenia (ang.
Parse Tree/Concrete Syntax Tree). Drzewo to odzwierciedla strukturę
naszego kodu; liśćmi w tym drzewie są tokeny dostarczone przez Lekser, a
gałęziami zasady składniowe, o których jeszcze sobie powiemy. W tym
etapie wykrywane są błędy składniowe (ang. syntax errors). Przejdźmy
teraz do przykładu działania parsera; chcielibyśmy aby nasz kompilator z
wyżej umieszczonego strumienia tokenów wygenerował następujące drzewo
wyprowadzenia:

\placefigure{Drzewo wyprowadzenia.}{\externalfigure[parseTree.png][width=7cm]}

Na powyższym drzewie wyprowadzenia dobrze widać strukturę naszego
programu; widać, które tokeny składają się na definicję funkcji
({\em funDef}), wywołanie funkcji ({\em funcCall}) oraz na operacje
matematyczne dodawania i mnożenia. Co więcej, z tego drzewa można
odczytać, że została zachowana poprawna kolejność wykonywania działań,
czyli mnożenie ma w naszym języku pierwszeństwo przed dodawaniem.

Drzewo wyprowadzenia po utworzeniu przekazywane jest do kolejnego kroku
kompilacji czyli analizy semantycznej. Podczas analizy semantycznej
przechodzimy po drzewie wyciągając z niego wszystkie informacje
potrzebne w dalszych krokach kompilacji.

Należy dodać, że wynikiem analizy składniowej bywa także dość często
nieco inna struktura danych: {\em drzewo składniowe} (ang. Abstract
Syntax Tree, w skrócie AST). Można powiedzieć, że drzewo składniowe jest
uproszczoną wersją drzewa wyprowadzenia. Nie powinny występować w nim
zbędne gałęzie i liście. W zasadzie zawiera ono tylko te informacje,
które są potrzebne na dalszym etapie kompilacji, a więc pomijane są w
nim np. informacje. o średnikach, klamerkach i nawiasach występujących w
kodzie źródłowym. Dodatkowo do AST podczas analizy semantycznej mogą być
dodawane nowe informacje. Drzewo wyprowadzenia można przekształcić do
drzewa składniowego, jednak my nie będziemy tego robić w ramach tego
artykułu.

\subsection[gramatyka-języka]{Gramatyka języka}

Na szczęście dla nas, ANTLR4 pozwala wygenerować kod Leksera i Parsera,
czyli części odpowiedzialnych za dwa pierwsze, wymienione wyżej kroki
kompilacji. Jednak najpierw musimy stworzyć plik z gramatyką naszego
języka, a dopiero z tego pliku ANTLR4 wygeneruje nam gotowy kod
Pythonowy. Plik z gramatyką naszego języka w zasadzie będzie zawierał 2
niżej wymienione gramatyki:

\startitemize[packed]
\item
  gramatyka leksykalna (zasady leksykalne czyli definicje tokenów),
  która jest swego rodzaju słownikiem naszego języka, ponieważ są w niej
  zapisane wszystkie słowa, których możemy w języku użyć
\item
  gramatyka składniowa (zasady składniowe), która definiuje składnię
  naszego języka, w niej opisana jest każda dozwolona w naszym języku
  konstrukcja, oraz np. to czy bloki kodu mają być wyznaczone wcięciami
  czy klamerkami, czy w naszym języku będą istnieć klasy, metody,
  funkcje, etc.
\stopitemize

Wróćmy do naszego pliku z gramatyką; powinien on nosić nazwę
\type{nazwa_naszego_języka.g4}, a pierwsza linia musi wyglądać~w
następujący sposób: \type{grammar nazwa_naszego_języka;}. W kolejnych
liniach pliku znajdują się zasady gramatyki w formie:

\starttyping
nazwa_zasady: opis_zasady;
\stoptyping

Zasady, których nazwy zaczynają się wielką literą są definicjami
tokenów, zaś pozostałe to zasady składniowe. Trzymając się dobrych
praktyk, zasady leksykalne nie powinny mieszać się z zasadami
składniowymi; natomiast mogą one znajdować się zarówno nad jak i pod
nimi. Jeżeli chodzi o opis zasady to może on się składać z wielu linii i
zawierać znane z wyrażeń regularnych konstrukcje takie jak grupowanie
{\em ( grupa )}, alternatywa {\em \letterbar{}}, kwantyfikatory
wielkościowe {\em *}, {\em +}, {\em ?} etc. Co istotne, kolejność zasad
leksykalnych (w przeciwieństwie do składniowych) ma duże znaczenie, co
postaram się poniżej wyjaśnić. Zobacz teraz jak będzie wyglądał plik z
gramatyką naszego wymyślonego języka:

Plik {\bf Conpy.g4}:

\starttyping
grammar Conpy;

FUNC: 'func';
IF: 'if';
ELSE: 'else';
INT: [0-9]+;
NAME: [a-zA-Z_][a-zA-Z_0-9]*;
WHITESPACE: [ \t]+ -> skip;
NEWLINE: ('\r' '\n'? | '\n') -> skip;

main: (funcDef | funcCall)* EOF;
funcDef: FUNC funcNameDef '(' paramDef? ')' block;
block: '{' (ifStmt | funcCall)* '}';
ifStmt: IF '(' expr ')' block (ELSE block)?;
funcCall: funcNameRef '(' expr? ')' ';';
expr: expr '*' expr | expr '+' expr | value;
value: paramRef | INT;
paramDef: NAME;
paramRef: NAME;
funcNameDef: NAME;
funcNameRef: NAME;
\stoptyping

Na początku pliku znajdują się zasady leksykalne, a następnie zasady
składniowe. Zaczniemy od omówienia zasad leksykalnych; każda z tych
zasad w swoim opisie zawiera informacje o tym z jakiego fragmentu kodu
źródłowego generuje token. Widzimy więc, że jeżeli w kodzie wystąpi
słowo {\em func} to zostanie wygenerowany token \type{FUNC}, a jeżeli
wystąpi np. słowo {\em PyConPL2019} to zostanie wygenerowany token
\type{NAME}. Dodatkowo dzięki tekstowi \type{-> skip}, dopisanemu na
końcach dwóch ostatnich definicji tokenów (\type{WHITESPACE},
\type{NEWLINE}), tokeny te nie zostaną dołączone do wynikowego
strumienia tokenów. Jest to zachowanie pożądane, gdyż jak już wcześniej
wspominałem, białe znaki oraz znaki nowej linii nie powinny mieć żadnego
wpływu (w naszym języku) na działanie programu. Zastanówmy się teraz,
jak będzie wyglądał strumień tokenów wygenerowany z poniższego kodu:

\starttyping
func functest
\stoptyping

Teoretyzując, w tym przypadku tokeny mogłyby się wygenerować na 2 różne
sposoby: \type{FUNC}, \type{FUNC}, \type{NAME}(ze słowa {\em test}) albo
\type{FUNC}, \type{NAME} (ze słowa {\em functest}). Zgadnij, które
tokeny zostaną wygenerowane? Zostanie wygenerowany drugi strumień
tokenów, czyli \type{FUNC}, \type{NAME}. Dlaczego? Podczas decyzji
leksera o tym, który token utworzy, decydujące są 2 czynniki, poza
oczywiście dopasowaniem fragmentu kodu do opisu zasady. Po pierwsze,
priorytet mają zasady, których opis dopasowuje dłuższy fragment kodu
źródłowego, a nie da się ukryć, że token \type{NAME} powstałby ze słowa
dłuższego ({\em testfunc} - 8 znaków) od tokenu \type{FUNC} ({\em func}
- 4 znaki). W przypadku gdy dopasowania do różnych zasad są równej
długości, priorytet ma zasada, która znajduje się wyżej, a więc ze słowa
{\em func} zostanie wygenerowany token \type{FUNC}, natomiast gdyby
definicję tokena \type{FUNC} przenieść na sam dół pliku to nigdy nie
zostałby on wygenerowany bo każde wystąpienie słowa {\em func}
generowałoby token \type{NAME}.

Można jeszcze zauważyć, że liczba zasad leksykalnych wydaje się zbyt
mała. W przykładzie działania leksera umieszczonym w sekcji {\em Analiza
leksykalna} widzimy, że zostały wygenerowane jeszcze następujące tokeny,
których definicji nie ma w naszej gramatyce: \type{OPEN_PAREN},
\type{CLOSE_PAREN}, \type{OPEN_BRACKET}, \type{CLOSE_BRACKET},
\type{PLUS}, \type{STAR}, \type{SEMICOLON}. Okazuje się, że te tokeny
zostaną wygenerowane, mimo, że nie są jawnie zdefiniowane. Dlaczego tak
się stanie powiemy sobie niżej, w trakcie wyjaśniania zasad gramatyki
składniowej.

Zasady gramatyki składniowej zaczynają się małą literą, a w ich opisie
możemy odwoływać się do zasad leksykalnych i składniowych oraz stosować
konstrukcje znane z wyrażeń regularnych. Do zasad leksykalnych / tokenów
można odwoływać się na dwa sposoby: przez nazwę albo przez opis.
Odwoływanie się przez nazwę to po prostu wszelkie wystąpienia słów typu
\type{EOF} (ang. end of file), \type{FUNC}, \type{IF}, \type{ELSE},
etc., natomiast odwoływanie się przez treść to np. wystąpienia:
\type{'*'} \type{'+'}, \mono{'\{'}, \mono{'\}'}, \type{'('}, \type{')'},
\type{';'} - w tym przypadku tokeny nie muszą być jawnie zdefiniowane,
gdyż zostaną one i tak automatycznie, niejawnie zdefiniowane. Jakie
zalety ma takie podejście? Kwestia gustu, można by wszystkie możliwe
tokeny zdefiniować jawnie i odwoływać się do nich przez nazwę (np.
\type{OPEN_BRACE} zamiast \mono{'\{'}) jednak według mnie gramatyka
składniowa jest czytelniejsza gdy do krótkich tokenów (jak np.
operatory) odwołujemy się przez treść.

No dobra, przejdźmy teraz do omówienia zasad leksykalnych naszej
gramatyki, zasada po zasadzie.

\startitemize[packed]
\item
  \type{main} ma reprezentować kod źródłowy pojedynczego pliku, mówi o
  tym, że w pliku może wystąpić zero lub więcej (kwantyfikator \type{*})
  definicji funkcji \type{funcDef} oraz wywołań funkcji \type{funcCall},
  a wszystko to zakończone znakiem \type{EOF} (znak końca pliku).
\item
  \type{funcDef} czyli definicja funkcji, składa się ze słowa kluczowego
  {\em func}, nazwy funkcji \type{funcNameDef}, nawiasu otwierającego,
  opcjonalnego (kwantyfikator \type{?}) parametru \type{paramDef},
  nawiasu zamykającego oraz bloku kodu \type{block}.
\item
  \type{block} reprezentujący blok kodu, składa się z klamerki
  otwierającej, następnie z zera lub więcej instrukcji \type{ifStmt} /
  \type{funcCall} oraz klamerki zamykającej.
\item
  \type{ifStmt} to instrukcja warunkowa if, składa się ze słowa
  {\em if}, nawiasu otwierającego, wyrażenia warunkowego \type{expr},
  nawiasu zamykającego, bloku kodu \type{block}, oraz na końcu może
  wystąpić opcjonalnie słowo {\em else} i kolejny blok kodu
  \type{block}.
\item
  \type{funcCall} czyli wywołanie funkcji, składa się z nazwy funkcji
  \type{funcNameRef}, nawiasu otwierającego, argumentu w formie
  wyrażenia \type{expr}, nawiasu zamykającego oraz średnika.
\item
  \type{expr} czyli wyrażenie, może być iloczynem dwóch wyrażeń
  \type{expr '*' expr}, sumą dwóch wyrażeń \type{expr '+' expr} lub
  wartością \type{value}.
\item
  \type{value} to nazwa będąca odwołaniem do parametru funkcji
  \type{paramRef} lub liczbą całkowitą.
\item
  \type{paramDef}, \type{paramRef}, \type{funcNameDef},
  \type{funcNameRef} oznaczają kolejno: parametr, odwołanie się do
  parametru, nazwę funkcji, odwołanie się do nazwy funkcji. Zasady te są
  nie są wymagane, można by je usunąć, a każdą z nich zastąpić
  odwołaniem do tokena \type{NAME}, natomiast mimo wszystko są
  przydatne, bo dzięki nim łatwiej nam będzie operować na wygenerowanym
  przez parser drzewie.
\stopitemize

\subsection[zaczynamy-pisać-w-pythonie]{Zaczynamy pisać w Pythonie!}

Jak już mamy napisaną gramatykę to teraz wystarczy skorzystać z ANTLR4,
aby wygenerować Lexer, Parser oraz dodatkowe klasy ułatwiające nam
napisać kod wykonujący kolejne kroki kompilacji.

\starttyping
antlr4 -Dlanguage=Python3 -visitor Conpy.g4
\stoptyping

Wykonujemy komendę \type{antlr4} wskazując, że językiem wynikowym ma być
\type{Python3}, dodatkowo dodajemy flagę \type{-visitor} aby wygenerować
klasę Visitora, a na koniec podajemy ścieżkę do pliku z gramatyką. Po
wykonanej komendzie w katalogu roboczym powinno pojawić się kilka
plików, z których istotne są w zasadzie tylko te z rozszerzeniem
\type{.py}:

\starttyping
ConpyLexer.py ConpyParser.py ConpyListener.py ConpyVisitor.py
\stoptyping

\type{ConpyLexer.py} zawiera klasę Leksera, \type{ConpyParser.py} klasę
Parsera, a \type{ConpyListener.py} i \type{ConpyVisitor.py} odpowiednio
Lisenera i Visitora, czyli klasy, które możemy wykorzystać do
łatwiejszego wyciągania danych z drzewa wyprowadzenia dostarczonego
przez Parser. Klasy te reprezentują w zasadzie dwa różne podejścia
korzystania z drzewa wyprowadzenia: Listener jak sama nazwa wskazuje
pozwala nasłuchiwać, gdy domyślne mechanizmy ANTLR4 przechodzą po
drzewie (za pomocą algorytmu BFS - przeszukiwania wszerz). Visitor zaś
to klasa, za pomocą której sami możemy zdecydować w jaki sposób będziemy
po drzewie chodzić. Klasa ta, dla każdej zasady gramatyki składniowej
posiada automatycznie wygenerowaną metodę \type{visitNazwaZasady} (ból
jest taki, że kod wygenerowany przez ANTLR4 stosuje konwencje
nazewnictwa zaczerpnięte z Javy), która jest wołana podczas
przechodzenia drzewa, w momencie wejścia do danej gałęzi. Metoda ta
dostaje w argumencie {\em kontekst} w którym zawarte są wszelkie
potrzebne informacje o danym węźle. Początek wygenerowanej klasy
\type{ConpyVisitor} prezentuje się następująco:

\starttyping
class ConpyVisitor(ParseTreeVisitor):
    def visitMain(self, ctx: ConpyParser.MainContext):
        return self.visitChildren(ctx)

    def visitFuncDef(self, ctx: ConpyParser.FuncDefContext):
        return self.visitChildren(ctx)

    def visitBlock(self, ctx: ConpyParser.BlockContext):
        return self.visitChildren(ctx)
    ...
\stoptyping

Domyślnie, każda z wygenerowanych dla zasad składniowych metod
(\type{visitNazwaZasady}) przechodzi dalej po wszystkich swoich dzieci
za pomocą metody \type{visitChildren} (która z kolei dla każdego dziecka
wywołuje \type{visitNazwaZasady}), natomiast nic nie stoi na
przeszkodzie, żeby zamiast tej metody {\em visit} wywołać jakąś inną.
Generalnie, chcąc skorzystać z Visitora nie powinno się edytować klasy
\type{ConpyVisitor}, a raczej stworzyć sobie własnego Visitora, który
będzie po niej dziedziczył i nadpisywał metody, których potrzebujemy - i
tak też zrobimy w dalszej części artykułu. Najpierw jednak zaczniemy od
stworzenia szkieletu naszego programu do którego później będziemy
dodawać kolejne fragmenty kodu:

\starttyping
import subprocess
import sys
import antlr4
from ConpyLexer import ConpyLexer
from ConpyParser import ConpyParser
from ConpyVisitor import ConpyVisitor as ConpyBaseVisitor

FuncDefContext = ConpyParser.FuncDefContext
FuncCallContext = ConpyParser.FuncCallContext

class SyntaxErrorListener(antlr4.DiagnosticErrorListener):
    def __init__(self, *args, **kwargs):
        super().__init__(*args, **kwargs)
        self.errors_count = 0

    def syntaxError(self, *args, **kwargs):
        self.errors_count += 1

def main():
    _, input_filename, output_filename = sys.argv

    input_stream = antlr4.FileStream(input_filename)
    lexer = ConpyLexer(input_stream)
    token_stream = antlr4.CommonTokenStream(lexer)
    parser = ConpyParser(token_stream)

    listener = SyntaxErrorListener()
    parser.addErrorListener(listener)
    parse_tree = parser.main()
    if listener.errors_count:
        msg = f"{listener.errors_count} syntax error generated"
        print(msg, file=sys.stderr)
        return

if __name__ == '__main__':
    main()
\stoptyping

Na początku importujemy wszystkie wymagane w tej i późniejszej chwili
pakiety i moduły. Następnie tworzymy dwa krótsze aliasy długich nazw ze
względu na to, że linie w których by one wystąpiły mogłyby przekroczyć
dopuszczalną szerokość (mniej niż 70 znaków na linie) tego artykułu.
Klasa \type{SyntaxErrorListener} to klasa dzięki której będziemy mogli
zliczyć błędy składniowe znalezione przez parser, a następnie zakończyć
program w przypadku ich wystąpienia. Przejdźmy teraz do funkcji
\type{main} czyli głównej funkcji naszego programu. Na początku
zakładam, że do programu są dostarczone dwie nazwy: pliku wejściowego z
kodem w języku {\em Conpy} oraz pliku wyjściowego z kodem wynikowym. Na
początku tworzymy Lekser do którego na wejściu przekazujemy zawartość
pliku wejściowego. Następnie tworzymy strumień tokenów, który
przekazujemy na wejściu do parsera. Następnie tworzymy nasz
\type{SyntaxErrorListener} i podpinamy do parsera. Parser proces analizy
składniowej / parsowania rozpoczyna dopiero w momencie wywołania metody
\type{main} (\type{main} to po prostu nazwa zasady z gramatyki
składniowej, równie dobrze moglibyśmy wywołać metodę \type{funcDef}), a
wynikiem jest \type{parse_tree} czyli drzewo wyprowadzenia. W trakcie
parsowania, na standardowy strumień błędów (stderr) zostają
automatycznie wypisane błędy składniowe. Na koniec sprawdzamy czy
wystąpiły jakieś błędy składniowe, a jeżeli tak to wypisujemy ile ich
było i kończymy program. Kolejny kod do naszego programu będziemy
dodawać w zasadzie w dwóch miejscach: nad funkcją \type{main} oraz na
końcu funkcji \type{main}, ale to już w kolejnym rozdziale\ldots{}

\subsection[analiza-semantyczna]{Analiza semantyczna}

Jest to etap wykonywany przez część kompilatora zwaną {\em analizatorem
semantycznym}. Głównym zadaniem tego etapu jest znajdowanie błędów
semantycznych, inaczej znaczeniowych. Są to takie błędy, o których nie
możemy stwierdzić, czy są błędami czy nie (bo składnia kodu jest
poprawna) bez znajomości szerszego kontekstu. Przykłady błędów
semantycznych:

\startitemize[packed]
\item
  przypisanie wartości do niezdefiniowanej zmiennej
\item
  dwukrotne zadeklarowanie tej samej nazwy w obrębie jednego bloku
\item
  niezgodność typów obiektów biorących udział w operacji matematycznej
  (np. dodawanie inta do stringa)
\item
  odwoływanie się do nieistniejącej zmiennej / atrybutu
\item
  odwoływanie się do indeksu spoza tablicy
\stopitemize

Takie błędy są zwykle wyłapywane przez kompilator właśnie podczas
analizy semantycznej. Python (CPython) jest tutaj swego rodzaju
wyjątkiem, gdyż ze względu na swoją dynamiczną naturę, bardzo trudne
bądź też niemożliwe byłoby wykrywanie tego typu błędów podczas
kompilacji do kodu bajtowego (py -> pyc). W przypadku Pythona można
jednak korzystać z adnotacji typów oraz narzędzi do statycznej analizy
kodu takich jak {\em mypy}, dzięki czemu część błędów uda się wyłapać
przed uruchomieniem programu.

Należy dodać, że podczas analizy semantycznej dane wyciągane z drzewa
często zapisywane są w różnych bardziej przyjemnych formach, z których
łatwiej można później wygenerować formę pośrednią lub kod wynikowy.

Co więcej, Analiza semantyczna również może (ale nie musi) się składać z
wielu kroków, przykładowe kroki takiej analizy to np. generowanie
tablicy symboli, rozwiązywanie zależności (dziedziczenie, złożone typy
zmiennych / atrybutów / funkcji), ewaluacja stałych, sprawdzanie
zgodności typów, etc. W przypadku naszego kompilatora, ze względu na
ograniczoną ilość miejsca przeznaczoną na ten artykuł, zdecydowałem, że
będzie to forma bardzo okrojona oraz skondensowana. Nasz analizator
składniowy będzie zaimplementowany w postaci Visitora oraz będzie
potrafił znaleźć kilka podstawowych błędów semantycznych takich jak:
liczba argumentów nie zgadzająca się z liczbą parametrów, podwójna
definicja funkcji o takiej samej nazwie czy odwoływanie się do
niezdefiniowanej wartości. Kod naszego analizatora składniowego
umieszczam poniżej:

\starttyping
 class SemanticAnalysisVisitor(ConpyBaseVisitor):
    def __init__(self):
        self.symbols_table = {"print": None}
        self.func_ctx = None
        self.errors_count = 0

    def perror(self, token, message):
        print(f"{token.line}:{token.column}: error: {message}",
              file=sys.stderr)
        self.errors_count += 1

    def visitFuncDef(self, ctx):
        self.func_ctx = ctx
        super().visitFuncDef(ctx)
        self.func_ctx = None

    def visitFuncNameDef(self, ctx):
        name, token = ctx.getText(), ctx.start
        if name in self.symbols_table:
            self.perror(token, f"redefined name '{name}'")
        else:
            self.symbols_table[name] = ctx.parentCtx

    def visitFuncNameRef(self, ctx):
        name, token = ctx.getText(), ctx.start
        if name not in self.symbols_table:
            self.perror(token, f"'{name}' undeclared")

    def visitParamDef(self, ctx):
        name = ctx.getText()
        func_name = self.func_ctx.funcNameDef().getText()
        self.symbols_table[f"{func_name}.{name}"] = ctx

    def visitParamRef(self, ctx):
        name, token = ctx.getText(), ctx.start
        if not self.func_ctx:
            self.perror(token, f"'{name}' undeclared")
            return
        func_name = self.func_ctx.funcNameDef().getText()
        param_name = f"{func_name}.{name}"
        if param_name not in self.symbols_table:
            self.perror(token, f"'{name}' undeclared")

    def visitFuncCall(self, ctx):
        super().visitFuncCall(ctx)
        name, token = ctx.start.text, ctx.start
        has_arg = bool(ctx.expr())
        if name not in self.symbols_table:
            return
        def_ctx, has_param = self.symbols_table[name], True
        if def_ctx:
            has_param = bool(def_ctx.paramDef().getText())
        if has_param is not has_arg:
            self.perror(token, "arguments count not match")
\stoptyping

Spójrzmy najpierw do metody \type{__init__}, tworzymy tam tablice
symboli \type{self.symbols_table} będącą słownikiem, w którym kluczami
będą nazwy zdefiniowanych funkcji i ich parametrów, a wartościami
{\em konteksty} gałęzi, z których one powstały. W idealnym scenariuszu
wartościami powinny być obiekty oddzielnej klasy (o nazwie np.
\type{Symbol}). Początkowo naszą tablicę symboli uzupełniłem nazwą
\type{"print"} ponieważ jest to nazwa zarezerwowana i nie chciałbym, aby
którykolwiek z symboli się tak nazywał. W zmiennej \type{self.func_ctx}
przechowywana będzie informacja o kontekście funkcji, w której ciele
aktualnie się znajdujemy (podczas przechodzenia drzewa). Oprócz tego
mamy również licznik błędów, który wykorzystamy w funkcji \type{main} do
zatrzymania programu jeżeli błędy wystąpiły. W metodach zaczynających
się od \type{visit} następuje wykrywanie błędów semantycznych (oprócz
\type{visitFuncDef}, która służy tylko do aktualizacji
\type{self.func_ctx}). Można zauważyć, że z tych metod tylko w metodzie
\type{visitFuncCall} wywołujemy bazową wersję metody, a to dlatego, że
tylko w tym przypadku da ona jakikolwiek efekt, ponieważ gałęzie
wygenerowane z zasad \type{funcNameDef}, \type{funcNameRef},
\type{paramDef}, \type{paramRef} nie mają podgałęzi które można
odwiedzić.

Warto również zwrócić uwagę na metodę \type{getText} kontekstu, która
zwraca połączony tekst ze wszystkich tokenów wchodzących w skład danej
gałęzi drzewa. Atrybut \type{start} kontekstu to po prostu pierwszy
token należący do węzła. Posiada on atrybuty takie jak \type{line} i
\type{column}, dzięki którym możemy odczytać gdzie w kodzie źródłowym
znajduje się fragment, z którego token został wygenerowany. Token ma
również atrybut \type{text} równoważny metodzie \type{getText}
kontekstu.

Aby skorzystać z naszego analizatora składniowego, do funkcji
\type{main}, na samym jej końcu, powinniśmy dodać następujący kod:

\starttyping
    visitor = SemanticAnalysisVisitor()
    visitor.visit(parse_tree)
    if visitor.errors_count:
        msg = f"{visitor.errors_count} semantic error generated"
        print(msg, file=sys.stderr)
        return
        
\stoptyping

\subsection[wygenerowanie-formy-pośredniej]{Wygenerowanie formy
pośredniej}

Kolejnym krokiem kompilacji jest generowanie formy pośredniej. Forma
pośrednia to pewna reprezentacja naszego programu, zapisana w formie
pewnych struktur danych lub częściej, w postaci języka programowania
(wtedy mówimy o języku pośrednim). Celem istnienia formy pośredniej jest
łatwiejsze wygenerowanie z niej kodu wynikowego. Językami pośrednimi
mogą być np. język asemblera (dla prawdziwej maszyny lub wirtualnej), C,
LLVM IR i wiele innych. W naszym kompilatorze ze względu na łatwość
generowania i niewielki rozmiar implementacji, zdecydowałem, że formą
pośrednią będzie kod języka C. Generator kodu, tak jak i analizator
semantyczny, będzie zaimplementowany w postaci visitora. Jego kod
możecie znaleźć poniżej i dodać nad funkcją \type{main}.

\starttyping
MAIN_TEMPLATE = """#include <stdio.h>
{} int main(void) {{ {} return 0; }}"""
IF_TEMPLATE = "if ({}) {{ {} }} else {{ {} }}"
FUNC_TEMPLATE = "void {} ({}) {{ {} }}"
CALL_TEMPLATE = "{} ({});"
PRINT_TEMPLATE = 'printf("%d\\n", {});'

class IRGenerationVisitor(ConpyBaseVisitor):
    def visitMain(self, ctx):
        defs, calls = [], []
        for child in ctx.children:
            if isinstance(child, FuncDefContext):
                defs.append(self.visitFuncDef(child))
            elif isinstance(child, FuncCallContext):
                calls.append(self.visitFuncCall(child))
        defs_code = '\n'.join(defs)
        calls_code = '\n'.join(calls)
        return MAIN_TEMPLATE.format(defs_code, calls_code)

    def visitFuncDef(self, ctx):
        name = ctx.funcNameDef().getText()
        param_code = ""
        if ctx.paramDef():
            param_code = f"int {ctx.paramDef().getText()}"
        block_code = self.visitBlock(ctx.block())
        return FUNC_TEMPLATE.format(name, param_code, block_code)

    def visitBlock(self, ctx):
        code = []
        for child in ctx.children:
            if isinstance(child, antlr4.TerminalNode):
                continue
            code.append(self.visit(child))
        return '\n'.join(code)

    def visitIfStmt(self, ctx):
        if_block, *else_block = ctx.block()
        if_code = self.visitBlock(if_block)
        else_code = ""
        if else_block:
            else_code = self.visitBlock(else_block[0])
        return IF_TEMPLATE.format(
            ctx.expr().getText(), if_code, else_code)

    def visitFuncCall(self, ctx):
        name_text = ctx.funcNameRef().getText()
        expr_text = ctx.expr().getText()
        if name_text == "print":
            return PRINT_TEMPLATE.format(expr_text)
        return CALL_TEMPLATE.format(name_text, expr_text)
\stoptyping

Na samym początku pliku znajdują się tekstowe szablony, przy pomocy
których będziemy generowali kod w języku C. W funkcji \type{visitMain}
korzystamy z atrybutu \type{children} kontekstu, który jest listą
kontekstów podrzędnych węzłów drzewa. Jako, że w języku C funkcje mogą
być wywoływane tylko wewnątrz innych funkcji, zaś definicje funkcji nie
mogą być zagnieżdżone (w naszym języku {\em Conpy} można mieszać
wywołania funkcji z definicjami), to teraz musimy je rozgraniczyć: kod
wygenerowany z wywołań funkcji dodajemy do jednej listy: \type{calls}, a
z definicji funkcji do drugiej: \type{defs}. Na koniec wszystko łączymy,
wrzucamy do szablonu \type{MAIN_TEMPLATE} i zwracamy.

Podobnie robimy w pozostałych metodach. Poruszę teraz jeszcze kwestie
warte wyjaśnienia. Otóż runtime ANTLR4 działa tak, że kontekst parsera
posiada metody o nazwach gałęzi podrzędnych danego węzła. Np. jako, że w
gramatyce, w zasadzie \type{funcDef} odwołujemy się do zasady
\type{block}, możemy w \type{visitFuncDef} wołać metodę
\type{funcDef.block} która zwróci nam kontekst podgałęzi \type{block}. W
przypadku jednak gdy w zasadzie składniowej odwołujemy się do innej
zasady więcej niż raz, np. w \type{ifStmt} dwa razy odwołujemy się do
zasady \type{block}, wtedy wywołanie \type{ctx.block} w metodzie
\type{visitIfStmt} zwróci nam listę kontekstów (o długości 1 lub 2, w
zależności czy wystąpił blok {\em else} czy nie).

Warto jeszcze przyjrzeć się zawartości metody \type{visitBlock}.
Występuje w niej sprawdzanie, czy kontekst dziecko jest węzłem kończącym
/ liściem (\type{antlr4.TerminalNode}), jeżeli tak to nie chcemy go
odwiedzać, gdyż odwiedzać można tylko gałęzie drzewa. Następnie, dla
gałęzi, odwiedzamy je za pomocą metody \type{self.visit}, która w
konsekwencji wykona odpowiednią metodę \type{visitNazwaZasady} w
zależności od przekazanego argumentu.

Koniec końców naszą formę pośrednią chcemy gdzieś zapisać. Aby to zrobić
na końcu funkcji \type{main} dodajemy następujący fragment kodu:

\starttyping
    visitor = IRGenerationVisitor()
    code = visitor.visit(parse_tree)
    filename_c = f"{input_filename}.c"
    with open(filename_c, "w") as file:
        file.write(code)
\stoptyping

\subsection[generowanie-kodu-wynikowego]{Generowanie kodu wynikowego}

W zasadzie, jeżeli naszą formą pośrednią jest język C to naszym kodem
wynikowym będzie kod maszynowy, którego wygenerowanie sprowadza się do
wywołania kompilatora języka C, np. {\em gcc}, który jest domyślnie
dostępny na systemie Linux. Aby to zrobić, na końcu funkcji \type{main}
należy dodać poniższy kod:

\starttyping
    process = subprocess.Popen(["gcc", "-x", "c", filename_c, "-o", output_filename])
    process.wait()
\stoptyping

Pełny kod programu można znaleźć pod adresem
\useURL[url4][http://github.com/Kisioj/ConpyCompiler][][github.com/Kisioj/ConpyCompiler]\from[url4]\footnote{Krzysztof
  Jura, \quotation{Conpy Compiler},
  \useURL[url5][https://github.com/Kisioj/ConpyCompiler]\from[url5]}.

Gdyby zaś naszą formą pośrednią był jakiś abstrakcyjny język asemblera,
przeznaczony dla maszyny wirtualnej, to kodem wynikowym byłby kod
bajtowy.

\subsection[optymalizacja]{Optymalizacja}

Na koniec warto napisać o optymalizacji, czyli kroku kompilacji często
wymienianym między {\em wygenerowaniem formy pośredniej}, a
{\em wygenerowaniem kodu wynikowego}. Często jednak optymalizacja odbywa
się na różnych etapach kompilacji. No dobra, ale co można
zoptymalizować? Przede wszystkim można powycinać z naszego programu
nieużywane funkcje i nieosiągalne fragmenty kodu (np. kod po return,
albo kod w bloku {\em if}, w którym warunek jest zawsze fałszywy). Można
wyliczyć wartości stałych oraz tych wyrażeń, które dadzą się wyliczyć na
etapie kompilacji. Dla funkcji {\em inline} można wstawić ich treść w
miejsce wywołania. Optymalizacje również często dotyczą kosztu operacji:
dzielenie przez potęgi dwójki zastępowane jest przez przesunięcia
bitowe. Więcej o optymalizacji można przeczytać na Wikipedii w artykule
\quotation{Optymalizacja kodu wynikowego} \footnote{Wikipedia,
  \quotation{Optymalizacja kodu wynikowego},
  \useURL[url6][https://pl.wikipedia.org/wiki/Optymalizacja_kodu_wynikowego]\from[url6]}.

W przypadku naszego kompilatora optymalizacja kodu wynikowego jest
robiona za nas przez kompilator {\em gcc}.

\subsection[źródła]{Źródła}

\startitemize[n,packed][stopper=.]
\item
  Wikipedia, \quotation{Kompilator},
  \useURL[url7][https://pl.wikipedia.org/wiki/Kompilator]\from[url7]
\item
  Krzysztof Jura, \quotation{Conpy Compiler},
  \useURL[url8][https://github.com/Kisioj/ConpyCompiler]\from[url8]
\item
  Wikipedia, \quotation{Optymalizacja kodu wynikowego},
  \useURL[url9][https://pl.wikipedia.org/wiki/Optymalizacja_kodu_wynikowego]\from[url9]
\item
  Terrence Par, \quotation{The Definitive ANTLR4 Reference}
\item
  Federico Tomassetti, \quotation{The ANTLR Mega Tutorial},
  \useURL[url10][https://tomassetti.me/antlr-mega-tutorial/]\from[url10]
\item
  Dzieje Khorinis, \quotation{Daedalus Compiler},
  \useURL[url11][https://github.com/dzieje-khorinis/DaedalusCompiler]\from[url11]
\stopitemize


\stoptext
